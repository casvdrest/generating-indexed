\documentclass[a4paper,msc,twosized=semi]{uustthesis}

\usepackage{framed}
\usepackage{mdframed}
\usepackage{setspace}
% \usepackage{extsizes}

\renewcommand{\figurename}{Listing}
\renewcommand{\listfigurename}{Code listings}

%% Listings 
\newenvironment{listing}[2] %% #1 = caption #2 = label
{
    \begin{figure}[h]
      \label{#2}
      \begin{framed}
        \caption{#1}
}
{
      \end{framed}
    \end{figure}
}

%% Agda snippets 
\newcommand{\includeagda}[2]{\vspace*{-0.35cm}\begin{center}\ExecuteMetaData[../src/chap0#1/latex/code.tex]{#2}\end{center}\vspace*{-0.35cm}}

%% Agda snippets, without removed spacing
\newcommand{\includeagdanv}[2]{\begin{center}\ExecuteMetaData[../src/chap0#1/latex/code.tex]{#2}\end{center}}

%% Agda snippets, not centered
\newcommand{\includeagdanc}[2]{\ExecuteMetaData[../src/chap0#1/latex/code.tex]{#2}\vspace*{-0.35cm}}

%% Agda listings
\newcommand{\includeagdalisting}[4]{
  \begin{listing}{#3}{#4} 
    \includeagdanc{#1}{#2}
  \end{listing} 
}

%% Agda snippets (appendices)
\newcommand{\appincludeagda}[2]{\ExecuteMetaData[../src/app#1/latex/code.tex]{#2}}

%% Agda listings (appendices)
\newcommand{\appincludeagdalisting}[4]{
  \begin{listing}{#3}{#4} 
    \appincludeagda{#1}{#2}
  \end{listing}
}

\newmdenv[
  topline=false,
  bottomline=false,
  rightline=false,
  skipabove=\topsep,
  skipbelow=\topsep
]{siderules}

\newenvironment{example}[0] 
{
  \begin{siderules}
    \vspace{-0.5cm}
    \paragraph{\textbf{Example}}
}
{
  \end{siderules}
}

%% ODER: format ==         = "\mathrel{==}"
%% ODER: format /=         = "\neq "
%
%
\makeatletter
\@ifundefined{lhs2tex.lhs2tex.sty.read}%
  {\@namedef{lhs2tex.lhs2tex.sty.read}{}%
   \newcommand\SkipToFmtEnd{}%
   \newcommand\EndFmtInput{}%
   \long\def\SkipToFmtEnd#1\EndFmtInput{}%
  }\SkipToFmtEnd

\newcommand\ReadOnlyOnce[1]{\@ifundefined{#1}{\@namedef{#1}{}}\SkipToFmtEnd}
\usepackage{amstext}
\usepackage{amssymb}
\usepackage{stmaryrd}
\DeclareFontFamily{OT1}{cmtex}{}
\DeclareFontShape{OT1}{cmtex}{m}{n}
  {<5><6><7><8>cmtex8
   <9>cmtex9
   <10><10.95><12><14.4><17.28><20.74><24.88>cmtex10}{}
\DeclareFontShape{OT1}{cmtex}{m}{it}
  {<-> ssub * cmtt/m/it}{}
\newcommand{\texfamily}{\fontfamily{cmtex}\selectfont}
\DeclareFontShape{OT1}{cmtt}{bx}{n}
  {<5><6><7><8>cmtt8
   <9>cmbtt9
   <10><10.95><12><14.4><17.28><20.74><24.88>cmbtt10}{}
\DeclareFontShape{OT1}{cmtex}{bx}{n}
  {<-> ssub * cmtt/bx/n}{}
\newcommand{\tex}[1]{\text{\texfamily#1}}	% NEU

\newcommand{\Sp}{\hskip.33334em\relax}


\newcommand{\Conid}[1]{\mathit{#1}}
\newcommand{\Varid}[1]{\mathit{#1}}
\newcommand{\anonymous}{\kern0.06em \vbox{\hrule\@width.5em}}
\newcommand{\plus}{\mathbin{+\!\!\!+}}
\newcommand{\bind}{\mathbin{>\!\!\!>\mkern-6.7mu=}}
\newcommand{\rbind}{\mathbin{=\mkern-6.7mu<\!\!\!<}}% suggested by Neil Mitchell
\newcommand{\sequ}{\mathbin{>\!\!\!>}}
\renewcommand{\leq}{\leqslant}
\renewcommand{\geq}{\geqslant}
\usepackage{polytable}

%mathindent has to be defined
\@ifundefined{mathindent}%
  {\newdimen\mathindent\mathindent\leftmargini}%
  {}%

\def\resethooks{%
  \global\let\SaveRestoreHook\empty
  \global\let\ColumnHook\empty}
\newcommand*{\savecolumns}[1][default]%
  {\g@addto@macro\SaveRestoreHook{\savecolumns[#1]}}
\newcommand*{\restorecolumns}[1][default]%
  {\g@addto@macro\SaveRestoreHook{\restorecolumns[#1]}}
\newcommand*{\aligncolumn}[2]%
  {\g@addto@macro\ColumnHook{\column{#1}{#2}}}

\resethooks

\newcommand{\onelinecommentchars}{\quad-{}- }
\newcommand{\commentbeginchars}{\enskip\{-}
\newcommand{\commentendchars}{-\}\enskip}

\newcommand{\visiblecomments}{%
  \let\onelinecomment=\onelinecommentchars
  \let\commentbegin=\commentbeginchars
  \let\commentend=\commentendchars}

\newcommand{\invisiblecomments}{%
  \let\onelinecomment=\empty
  \let\commentbegin=\empty
  \let\commentend=\empty}

\visiblecomments

\newlength{\blanklineskip}
\setlength{\blanklineskip}{0.66084ex}

\newcommand{\hsindent}[1]{\quad}% default is fixed indentation
\let\hspre\empty
\let\hspost\empty
\newcommand{\NB}{\textbf{NB}}
\newcommand{\Todo}[1]{$\langle$\textbf{To do:}~#1$\rangle$}

\EndFmtInput
\makeatother
%
%
%
%
%
%
% This package provides two environments suitable to take the place
% of hscode, called "plainhscode" and "arrayhscode". 
%
% The plain environment surrounds each code block by vertical space,
% and it uses \abovedisplayskip and \belowdisplayskip to get spacing
% similar to formulas. Note that if these dimensions are changed,
% the spacing around displayed math formulas changes as well.
% All code is indented using \leftskip.
%
% Changed 19.08.2004 to reflect changes in colorcode. Should work with
% CodeGroup.sty.
%
\ReadOnlyOnce{polycode.fmt}%
\makeatletter

\newcommand{\hsnewpar}[1]%
  {{\parskip=0pt\parindent=0pt\par\vskip #1\noindent}}

% can be used, for instance, to redefine the code size, by setting the
% command to \small or something alike
\newcommand{\hscodestyle}{}

% The command \sethscode can be used to switch the code formatting
% behaviour by mapping the hscode environment in the subst directive
% to a new LaTeX environment.

\newcommand{\sethscode}[1]%
  {\expandafter\let\expandafter\hscode\csname #1\endcsname
   \expandafter\let\expandafter\endhscode\csname end#1\endcsname}

% "compatibility" mode restores the non-polycode.fmt layout.

\newenvironment{compathscode}%
  {\par\noindent
   \advance\leftskip\mathindent
   \hscodestyle
   \let\\=\@normalcr
   \let\hspre\(\let\hspost\)%
   \pboxed}%
  {\endpboxed\)%
   \par\noindent
   \ignorespacesafterend}

\newcommand{\compaths}{\sethscode{compathscode}}

% "plain" mode is the proposed default.
% It should now work with \centering.
% This required some changes. The old version
% is still available for reference as oldplainhscode.

\newenvironment{plainhscode}%
  {\hsnewpar\abovedisplayskip
   \advance\leftskip\mathindent
   \hscodestyle
   \let\hspre\(\let\hspost\)%
   \pboxed}%
  {\endpboxed%
   \hsnewpar\belowdisplayskip
   \ignorespacesafterend}

\newenvironment{oldplainhscode}%
  {\hsnewpar\abovedisplayskip
   \advance\leftskip\mathindent
   \hscodestyle
   \let\\=\@normalcr
   \(\pboxed}%
  {\endpboxed\)%
   \hsnewpar\belowdisplayskip
   \ignorespacesafterend}

% Here, we make plainhscode the default environment.

\newcommand{\plainhs}{\sethscode{plainhscode}}
\newcommand{\oldplainhs}{\sethscode{oldplainhscode}}
\plainhs

% The arrayhscode is like plain, but makes use of polytable's
% parray environment which disallows page breaks in code blocks.

\newenvironment{arrayhscode}%
  {\hsnewpar\abovedisplayskip
   \advance\leftskip\mathindent
   \hscodestyle
   \let\\=\@normalcr
   \(\parray}%
  {\endparray\)%
   \hsnewpar\belowdisplayskip
   \ignorespacesafterend}

\newcommand{\arrayhs}{\sethscode{arrayhscode}}

% The mathhscode environment also makes use of polytable's parray 
% environment. It is supposed to be used only inside math mode 
% (I used it to typeset the type rules in my thesis).

\newenvironment{mathhscode}%
  {\parray}{\endparray}

\newcommand{\mathhs}{\sethscode{mathhscode}}

% texths is similar to mathhs, but works in text mode.

\newenvironment{texthscode}%
  {\(\parray}{\endparray\)}

\newcommand{\texths}{\sethscode{texthscode}}

% The framed environment places code in a framed box.

\def\codeframewidth{\arrayrulewidth}
\RequirePackage{calc}

\newenvironment{framedhscode}%
  {\parskip=\abovedisplayskip\par\noindent
   \hscodestyle
   \arrayrulewidth=\codeframewidth
   \tabular{@{}|p{\linewidth-2\arraycolsep-2\arrayrulewidth-2pt}|@{}}%
   \hline\framedhslinecorrect\\{-1.5ex}%
   \let\endoflinesave=\\
   \let\\=\@normalcr
   \(\pboxed}%
  {\endpboxed\)%
   \framedhslinecorrect\endoflinesave{.5ex}\hline
   \endtabular
   \parskip=\belowdisplayskip\par\noindent
   \ignorespacesafterend}

\newcommand{\framedhslinecorrect}[2]%
  {#1[#2]}

\newcommand{\framedhs}{\sethscode{framedhscode}}

% The inlinehscode environment is an experimental environment
% that can be used to typeset displayed code inline.

\newenvironment{inlinehscode}%
  {\(\def\column##1##2{}%
   \let\>\undefined\let\<\undefined\let\\\undefined
   \newcommand\>[1][]{}\newcommand\<[1][]{}\newcommand\\[1][]{}%
   \def\fromto##1##2##3{##3}%
   \def\nextline{}}{\) }%

\newcommand{\inlinehs}{\sethscode{inlinehscode}}

% The joincode environment is a separate environment that
% can be used to surround and thereby connect multiple code
% blocks.

\newenvironment{joincode}%
  {\let\orighscode=\hscode
   \let\origendhscode=\endhscode
   \def\endhscode{\def\hscode{\endgroup\def\@currenvir{hscode}\\}\begingroup}
   %\let\SaveRestoreHook=\empty
   %\let\ColumnHook=\empty
   %\let\resethooks=\empty
   \orighscode\def\hscode{\endgroup\def\@currenvir{hscode}}}%
  {\origendhscode
   \global\let\hscode=\orighscode
   \global\let\endhscode=\origendhscode}%

\makeatother
\EndFmtInput
%
%
%
\ReadOnlyOnce{colorcode.fmt}%

\RequirePackage{colortbl}
\RequirePackage{calc}

\makeatletter
\newenvironment{colorhscode}%
  {\hsnewpar\abovedisplayskip
   \hscodestyle
   \tabular{@{}>{\columncolor{codecolor}}p{\linewidth}@{}}%
   \let\\=\@normalcr
   \(\pboxed}%
  {\endpboxed\)%
   \endtabular
   \hsnewpar\belowdisplayskip
   \ignorespacesafterend}

\newenvironment{tightcolorhscode}%
  {\hsnewpar\abovedisplayskip
   \hscodestyle
   \tabular{@{}>{\columncolor{codecolor}\(}l<{\)}@{}}%
   \pmboxed}%
  {\endpmboxed%
   \endtabular
   \hsnewpar\belowdisplayskip
   \ignorespacesafterend}

\newenvironment{barhscode}%
  {\hsnewpar\abovedisplayskip
   \hscodestyle
   \arrayrulecolor{codecolor}%
   \arrayrulewidth=\coderulewidth
   \tabular{|p{\linewidth-\arrayrulewidth-\tabcolsep}@{}}%
   \let\\=\@normalcr
   \(\pboxed}%
  {\endpboxed\)%
   \endtabular
   \hsnewpar\belowdisplayskip
   \ignorespacesafterend}
\makeatother

\def\colorcode{\columncolor{codecolor}}
\definecolor{codecolor}{rgb}{1,1,.667}
\newlength{\coderulewidth}
\setlength{\coderulewidth}{3pt}

\newcommand{\colorhs}{\sethscode{colorhscode}}
\newcommand{\tightcolorhs}{\sethscode{tightcolorhscode}}
\newcommand{\barhs}{\sethscode{barhscode}}

\EndFmtInput

%%%%%%%%%%%%%%%%%%%%%%%%%%%%%%
%% 
%% Haskell Styling
%%
%% TODO: Figure out spacing!

%% Colors (from duo-tone light syntax)
\definecolor{hsblack}{RGB}{45,32,3}
\definecolor{hsgold1}{RGB}{179,169,149}
\definecolor{hsgold2}{RGB}{177,149,90}
\definecolor{hsgold3}{RGB}{190,106,13}%{192,96,4}%{132,97,19}
\definecolor{hsblue1}{RGB}{173,176,182}
\definecolor{hsblue2}{RGB}{113,142,205}
\definecolor{hsblue3}{RGB}{0,33,132}
\definecolor{hsblue4}{RGB}{97,108,132}
\definecolor{hsblue5}{RGB}{34,50,68}
\definecolor{hsred2}{RGB}{191,121,103}
\definecolor{hsred3}{RGB}{171,72,46}

%% LaTeX Kerning nastiness. By using curly braces to delimit color group,
%% it breaks spacing. The following seems to work:
%%
%% https://tex.stackexchange.com/questions/85033/colored-symbols/85035#85035
%%
\newcommand*{\mathcolor}{}
\def\mathcolor#1#{\mathcoloraux{#1}}
\newcommand*{\mathcoloraux}[3]{%
  \protect\leavevmode
  \begingroup
    \color#1{#2}#3%
  \endgroup
}
\newcommand{\HSKeyword}[1]{\mathcolor{hsgold3}{\textbf{#1}}}
\newcommand{\HSNumeral}[1]{\mathcolor{hsred3}{#1}}
\newcommand{\HSChar}[1]{\mathcolor{hsred2}{#1}}
\newcommand{\HSString}[1]{\mathcolor{hsred2}{#1}}
\newcommand{\HSSpecial}[1]{\mathcolor{hsblue4}{#1}}
\newcommand{\HSSym}[1]{\mathcolor{hsblue4}{#1}}
\newcommand{\HSCon}[1]{\mathcolor{hsblue3}{\mathit{#1}}}
\newcommand{\HSVar}[1]{\mathcolor{hsblue5}{\mathit{#1}}}
\newcommand{\HSComment}[1]{\mathcolor{hsgold2}{\textit{#1}}}


%%% lhs2TeX parser does not recognize '*' 
%%% in kind annotations, it thinks it is a multiplication.



\usepackage{ucs}
\usepackage[utf8x]{inputenc}
\usepackage{autofe}
\usepackage{textcomp}

%% Haskell snippet 
\newenvironment{myhaskell}
{
  \vspace{-0.35cm}
  \begin{center}
}
{
  \end{center}
  \vspace{-0.35cm}
}

%% Haskell snippet 
\newenvironment{myhaskellnv}
{
  \begin{center}
}
{
  \end{center}
}


\title{Thesis title}

\author{C.R. van der Rest}

\supervisor{Dr. W.S. Swierstra \\ Dr. M.M.T. Chakravarty \\ Dr. A. Serrano Mena }

\begin{document}
\maketitle

%% Set up the front matter of our book
\frontmatter
\tableofcontents

\chapter{Declaration}
Thanks to family, supervisor, friends and hops!

\chapter{Abstract}
Abstract

%% Starts the mainmatter
\mainmatter

\chapter{Introduction}

\chapter{Background}
In this section, we will briefly discuss some of the relevant theoretical background 
for this thesis. We assume the reader to be familiar with the general concepts of both 
Haskell and Agda, as well as functional programming in general. We shortly touch upon 
the following subjects:

\begin{itemize}
  \item
  \emph{Type theory} and its relationship with \emph{classical logic} through the 
  \emph{Curry-Howard correspondence}

  \item 
  Some of the more advanced features of the programming language \emph{Agda}, which we 
  use for the formalization of our ideas: \emph{Codata}, \emph{Sized Types} and \emph
  {Universe Polymorphism}. 

  \item 
  \emph{Datatype generic programming} using \emph{type universes} and the design 
  patterns associated with datatype generic programming.  
\end{itemize}

  We present this section as a courtesy to those readers who might not be familiar 
  with these topics; anyone experienced in these areas should feel free to skip ahead. 

\section{Type Theory}

  \emph{Type theory} is the mathematical foundation that underlies the \emph{type 
  systems} of many modern programming languages. In type theory, we reason about \emph\
  {terms} and their \emph{types}. We briefly introduce some basic concepts, and show 
  how they relate to our proofs in Agda. 

  \subsection{Intuitionistic Type Theory}

  In Intuitionistic type theory consists of terms, types and judgements $a : A$ 
  stating that terms have a certain type. Generally we have the following two finite 
  constructions: $\mathbb{0}$ or the \emph{empty type}, containing no terms, and 
  $\mathbb{1}$ or the \emph{unit type} which contains exactly $1$ term. Additionally,
  the \emph{equality type}, $=$, captures the notion of equality for both terms and 
  types. The equalit type is constructed from \emph{reflexivity}, i.e. it is 
  inhabited by one term $refl$ of the type $a = a$. 

  Types may be combined using three constructions. The \emph{function type}, $a 
  \rightarrow b$ is inhabited by functions that take an element of type $a$ as input 
  and produce something of type $b$. The \emph{sum type}, $a + b$ creates a type that 
  is inhabited by \emph{either} a value of type $a$ \emph{or} a 
  value of type $b$. The \emph{product type}, $a * b$, is inhabited by a pair of 
  values, one of type $a$ and one of type $b$. In terms of set theory, these 
  operations correspond respectively to functions, \emph{cartesian product} and \emph
  {tagged union}. 

  \subsection{The Curry-Howard Equivalence}

  The \emph{Curry-Howard equivalence} establishes an isomorphism between \emph
  {propositions and types} and \emph{proofs and terms} \cite{wadler2015propositions}. 
  This means that for any type there is a corresponding proposition, and any term 
  inhabiting this type corresponds to a proof of the associated proposition. Types and 
  propositions are generally connected using the mapping shown in \cref{tbl:chiso}.

\begin{table}[h]\label{tbl:chiso}
\begin{center}\begin{framed}
\begin{tabular}{ll}
\multicolumn{1}{c}{\textbf{Classical Logic}} & \textbf{Type Theory} \\ \hline \hline
False                                        & $\bot$               \\
True                                         & $\top$               \\
$P \vee Q$                                   & $P + Q$              \\
$P \wedge Q$                                 & $P * Q$              \\
$p \Rightarrow Q$                            & $P \rightarrow Q$                       
\end{tabular}
\caption{Correspondence between classical logic and type theory}
\end{framed}\end{center}
\end{table}

  \begin{example}

    We can model the proposition $P \wedge (Q \vee R) \Rightarrow (P \wedge Q) \vee (P 
    \wedge R)$ as a function with the following type: 

\includeagdanv{2}{tautologytype}

    We can then prove that this implication holds on any proposition by supplying a 
    definition that inhabits the above type: 

\includeagda{2}{tautologydef}

  \end{example}

  In general, we may prove any proposition that captured as a type by writing a 
  programin that inhabits that type. Allmost all constructs are readily translatable 
  from proposition logic, except boolean negation, for which there is no corresponding 
  construction in type theory. Instead, we model negation using functions to the empty 
  type $\bot$. That is, we can prove a property $P$ to be false by writing a function \
  $P \rightarrow \bot$. This essentially says that $P$ is true, we can derive a \
  contradiction, hence it must be false. Alowing us to prove many properties including negation. 
  
  \begin{example}

    For example, we might prove that a property 
    cannot be both true and false, i.e. $\forall\ P\ .\ \neg(P \wedge \neg P)$: 

\includeagdanv{2}{notpandnotp}

  \end{example}

  However, there are properties of classical logic which do not carry over well 
  through the Curry-Howard isomorphism. A good example of this is the \emph{law of 
  excluded middle}, which cannot be proven in type theory: 

\includeagda{2}{excludedmiddle}

  This implies that type theory is incomplete as a proof system, in the sense that 
  there exist properties wich we cannot prove, nor disprove. 

\subsection{Dependent Types}

  Dependent type theory allows the definition of types that depend on values. In 
  addition to the constructs introduced above, one can use so-called $\Pi$-types and 
  $\Sigma$-types. 
  $\Pi$-types capture the idea of \emph{dependent function types}, that is, functions 
  whose output type may depend on the values of its input. Given some type $A$ and a 
  family $P$ of types indexed by values of type $A$ (i.e. $P$ has type $A \rightarrow 
  Type$), $\Pi$-types have the following form: 

\begin{equation*}
\Pi_{(x : A)} P(x) \equiv (x : A) \rightarrow P(x) 
\end{equation*}

  In a similar spirit, $\Sigma$-types are ordered \textit{pairs} of which the type
  of the second value may depend on te first value of the pair:

\begin{equation*}
\Sigma_{(x : A)} P(x) \equiv (x : A) \times P(x) 
\end{equation*}

  The Curry-Howard equivalence extends to $\Pi$- and $\Sigma$-types as well: they 
  can be used to model universal and existential quantification \cite
  {wadler2015propositions} (\cref{chisodependent}).

\begin{table}[h]\label{tbl:chisodependent}
\begin{center}\begin{framed}
\begin{tabular}{ll}
\multicolumn{1}{c}{\textbf{Classical Logic}} & \textbf{Type Theory}    \\ \hline \hline
$\exists\ x\ .\ P\ x$                        & $\Sigma_{(x : A)} P(x)$ \\
$\forall\ x\ .\ P\ x$                        & $\Pi_{(x : A)} P(x)$                    
\end{tabular}
\caption{Correspondence between quantifiers in classical logic and type theory}
\end{framed}\end{center}
\end{table}

  \begin{example} 
  
    we might capture the relation between universal and negated existential 
    quantification ($\forall\ x\ .\ \neg P\ x \Rightarrow \neg \exists\ x\ .\ P\ x$) 
    as follows: 

\includeagdanv{2}{forallnottonotexists} 

  \end{example}

  The correspondence between dependent pairs and existential quantification quite \
  beautifullly illustrates the constructive nature of proofs in type theory; we prove 
  any existential property by presenting a term together with a proof that the 
  required property holds for that term. 

\section{Agda}

  Agda is a programming language based on Intuitionistic type theory\cite
  {norell2008dependently}. Its syntax is broadly similar to Haskell's, though Agda's 
  type system is arguably more expressive, since types may depend on term level 
  values. 

  Due to the aforementioned correspondence between types and propositions, any Agda 
  program we write is simultaneously a proof of the proposition associated with its 
  type. Through this mechanism, Agda serves a dual purpose as a proof assistent. 

\subsection{Codata and Sized Types}\label{codata}

  All definitions in Agda are required to be \textit{total}, meaning that they must be 
  defined on all possible inputs, produce a result in finite time. To enforce this 
  requirement, Agda needs to check whether the definitions we provide are terminating. 
  As stated by the \emph{Halting Problem}, it is not possible to define a general 
  procedure to perform this check. Instead, Agda uses a \emph{sound approximation}, in 
  which it requires at least one argument of any recursive call to be \emph
  {syntactically smaller} than its corresponding function argument. A consequence of 
  this approach is that there are Agda programs that terminate, but are rejected by 
  the termination checker. This means that we cannot work with infinite data in the 
  same way as in the same way as in Haskell, which does not care about termination. 
  
  \begin{example}

    The following definition is perfectly fine in Haskell: 

\begin{myhaskellnv}
\begin{hscode}\SaveRestoreHook
\column{B}{@{}>{\hspre}l<{\hspost}@{}}%
\column{E}{@{}>{\hspre}l<{\hspost}@{}}%
\>[B]{}\HSVar{nats}\HSSym{::}\HSSpecial{\HSSym{[\mskip1.5mu} }\HSCon{Int}\HSSpecial{\HSSym{\mskip1.5mu]}}{}\<[E]%
\\
\>[B]{}\HSVar{nats}\HSSym{\mathrel{=}}\HSNumeral{0}\HSCon{\mathbin{:}}\HSVar{map}\;\HSSpecial{(}\HSSym{+}\HSNumeral{1}\HSSpecial{)}\;\HSVar{nats}{}\<[E]%
\ColumnHook
\end{hscode}\resethooks
\end{myhaskellnv}

    Meanwhile, an equivalent definition in Agda gets rejected by the Termination 
    checker. The recursive call to \ensuremath{\HSVar{nats}} has no arguments, so none are structurally 
    smaller, thus the termination checker flags this call.  

\includeagda{2}{natsnonterminating}

  \end{example}

  However, as long as we use \ensuremath{\HSVar{nats}} sensibly, there does not need to be a problem. 
  Nonterminating programs only arise with improper use of such a definition, for 
  example by calculating the length of \ensuremath{\HSVar{nats}}. We can prevent the termination 
  checker from flagging these kind of operations by making the lazy semantics 
  explicit, using \textit{codata} and {sized types}. Codata is a general term for 
  possible inifinite data, often described by a co-recursive definition. Sized types 
  extend the space of function definitions that are recognized by the termination 
  checker as terminating by tracking information about the size of values in types 
  \cite{abel2010miniagda}. In the case of lists, this means that we explicitly 
  specify that the recursive argument to the \ensuremath{\HSSym{\anonymous} \HSSym{∷\char95 }} constructor is a \textit{Thunk}, 
  which should only be evaluated when needed: 

\includeagda{2}{colist}

  We can now define \ensuremath{\HSVar{nats}} in Agda by wrapping the recursive call in a thunk, 
  explicitly marking that it is not to be evaluated until needed.  

\includeagda{2}{natsterminating}

  Since colists are possible infinite structures, there are some functions we can 
  define on lists, but not on colists. 
    
  \begin{example} Consider a function that attempts to calculate the length of a \ensuremath{\HSCon{Colist}}: 

\includeagdanv{2}{lengthdef}

    In this case \ensuremath{\HSVar{length}} is not accepted by the termination checker because the input 
    colist is indexed with size \ensuremath{\HSSym{∞}}, meaning that there is no finite upper bound on 
    its size. Hence, there is no guarantee that our function terminates when 
    inductively defined on the input colist. 

  \end{example}
  
  There are some cases in which we can convince the termination checker that our definition is terminating by using sized types. Consider the folowing function that increments every element in a list of naturals with its position: 

\includeagda{2}{incposdef}

  The recursive call to \ensuremath{\HSVar{incpos}} gets flagged by the termination checker; we know 
  that \ensuremath{\HSVar{map}} does not alter the length of a list, but the termination checker cannot 
  see this. For all it knows \ensuremath{\HSVar{map}} equals \ensuremath{\HSVar{const}\;\HSSpecial{\HSSym{[\mskip1.5mu} }\HSNumeral{1}\HSSpecial{\HSSym{\mskip1.5mu]}}}, which would make \ensuremath{\HSVar{incpos}} 
  non-terminating. The size-preserving property of \ensuremath{\HSVar{map}} is not reflected in its 
  type. To mitigate this issue, we can define an alternative version of the \ensuremath{\HSCon{List}} 
  datatype indexed with \ensuremath{\HSCon{Size}}, which tracks the depth of a value in its type. 

\includeagda{2}{sizedlistdef}

  Here \ensuremath{\HSSym{↑}\HSVar{i}} means that the depth of a value constructed using the $::$ constructor 
  is one deeper than its recursive argument. Incidently, the recursive depth of a 
  list is equal to its size (or length), but this is not necessarily the case. By 
  indexing values of \ensuremath{\HSCon{List}} with their size, we can define a version of \ensuremath{\HSVar{map}} which 
  reflects in its type that the size of the input argument is preserved: 

\includeagda{2}{sizedmapdef}

  Using this definition of \ensuremath{\HSVar{map}}, the definition of \ensuremath{\HSVar{incpos}} is no longer rejected 
  by the termination checker. 

\subsection{Universe Polymorphism}

  Contrary to Haskell, Agda does not have separate notions for \emph{types}, 
  \emph{kinds} and \emph{sorts}. Instead it provides an infinite hierarchy of 
  type universes, where level is a member of the next, i.e. \ensuremath{\HSCon{Set}\;\HSVar{n}\HSCon{\mathbin{:}}\HSCon{Set}\;\HSSpecial{(}\HSVar{n}\HSSym{+}\HSNumeral{1}\HSSpecial{)}}. 
  Agda uses this construction in favor of simply declaring \ensuremath{\HSCon{Set}\HSCon{\mathbin{:}}\HSCon{Set}} to avoid 
  the construction of contradictory statements through Russel's paradox. 

  This implies that every construction in Agda that ranges over some \ensuremath{\HSCon{Set}\;\HSVar{n}} can 
  only be used for values that are in \ensuremath{\HSCon{Set}\;\HSVar{n}}. It is not possible to define, for 
  example, a \ensuremath{\HSCon{List}} datatype that may contain both \emph{values} and \emph{types}
   for this reason. 

   We can work around this limitation by defining a \emph{universe polymorphic} 
   construction for lists: 

\includeagda{2}{upolylist}

  Allowing us to capture lists of types (such as \ensuremath{\HSCon{ℕ}\HSSym{∷}\HSCon{Bool}\HSSym{∷}\HSSpecial{\HSSym{[\mskip1.5mu} }\HSSpecial{\HSSym{\mskip1.5mu]}}}) and lists of 
  values (such as \ensuremath{\HSNumeral{1}\HSSym{∷}\HSNumeral{2}\HSSym{∷}\HSSpecial{\HSSym{[\mskip1.5mu} }\HSSpecial{\HSSym{\mskip1.5mu]}}}) using a single datatype. Agda allows for the 
  programmer to declare that \ensuremath{\HSCon{Set}\HSCon{\mathbin{:}}\HSCon{Set}} using the \ensuremath{\mbox{\enskip\{-\# OPTIONS --type-in-type  \#-\}\enskip}} 
  pragma. Throughout the development accompanying this thesis, we will refrain from 
  using this pragma wherever possible. The examples included in this thesis are often 
  not universe-polymorphic, since the universe level variables required often pollute 
  the code, and obfuscate the concept we are trying to convey. 

\section{Generic Programming and Type Universes}

  In \emph{Datatype generic programming}, we define functionality not for individual 
  types, but rather by induction on \emph{structure} of types. This means that generic 
  functions will not take values of a particular type as input, but a \emph{code} that 
  describes the structure of a type. Haskell's \ensuremath{\HSKeyword{deriving}} mechanism is a prime example 
  of this mechanism. Anytime we add \ensuremath{\HSKeyword{deriving}\;\HSCon{Eq}} to a datatype definition, GHC will, 
  in the background, convert our datatype to a structural representation, and use a 
  \emph{generic equality} to create 
  an instance of the \ensuremath{\HSCon{Eq}} typeclass for our type. 

\subsection{Design Pattern}\label{sec:tudesignpattern}

  Datatype generic programming often follows a common design pattern that is 
  independent of the structural representation of types involved. In general 
  we follow the following steps: 

  \begin{enumerate}
    \item
      First, we define a datatype \ensuremath{\HSCon{𝓤}} representing the structure of types, 
      often called a \emph{Universe}. 
    \item 
      Next, we define a semantics \ensuremath{\HSSym{⟦\char95 ⟧}\HSCon{\mathbin{:}}\HSCon{𝓤}\HSSym{→}\HSCon{K}} that associates codes in \ensuremath{\HSCon{𝓤}} 
      with an appropriate value of kind \ensuremath{\HSCon{K}}. In practice this is often a functorial 
      representation of kind \ensuremath{\HSCon{Set}\HSSym{→}\HSCon{Set}}.
    \item 
      Finally, we (often) define a fixed point combinator of type \ensuremath{\HSSpecial{(}\HSVar{u}\HSCon{\mathbin{:}}\HSCon{𝓤}\HSSpecial{)}\HSSym{→}\HSCon{Set}} 
      that calculates the fixpoint of \ensuremath{\HSSym{⟦}\HSVar{u}\HSSym{⟧}}. 
  \end{enumerate}

  This imposes the implicit requirement that if we want to represent some type 
  \ensuremath{\HSCon{T}} with a code \ensuremath{\HSVar{u}\HSCon{\mathbin{:}}\HSCon{𝓤}}, the fixpoint of \ensuremath{\HSVar{u}} should be isomorphic to \ensuremath{\HSCon{T}}. 

  Given these ingredients we have everything we need at hand to write generic 
  functions. Section $3$ of Ulf Norell's \emph{Dependently Typed Programming 
  in Agda} \cite{norell2008dependently} contains an in depth explanation of 
  how this can be done in Agda. We will only give a rough sketch of the most 
  common design pattern here. In general, a datatype generic function is supplied
  with a code \ensuremath{\HSVar{u}\HSCon{\mathbin{:}}\HSCon{𝓤}}, and returns a function whose type is dependent on the 
  code it was supplied with. 
  
  \begin{example}

    Suppose we are defining a generic procedure for decidable equality. We might use the following type signature for such a procedure:

\includeagdanv{2}{eqdef}

    If we now define \ensuremath{\HSSym{≟}} by induction over \ensuremath{\HSVar{u}}, we have a decision procedure 
    for decidable equality that may act on values on any type, provided their 
    structure can be described as a code in \ensuremath{\HSCon{𝓤}}. 

  \end{example}

\subsection{Example Universes}

  There exist many different type universes. We will give a short overview of 
  the universes used in this thesis here; they will be explained in more detail 
  later on when we define generic generators for them. The literature review in 
  \cref{sec:lituniverses} contains a brief discussion of type universes beyond 
  those used we used for generic enumeration. 

  \paragraph{Regular Types} 
    Although the universe of regular types is arguably 
    one of the simplest type universes, it can describe a wide variaty of 
    recursive algebraic datatypes \texttt{[citation]}, roughly corresponding to 
    the algebraic types in Haskell98. Examples of regular types are 
    \emph{natural numbers}, \emph{lists} and \emph{binary trees}. Regular types are 
    insufficient once we want to have a generic representation of mutually recursive 
    or indexed datatypes. 

  \paragraph{Indexed Containers}
    The universe of \emph{Indexed Containers} \cite{altenkirch2015indexed} 
    provides a generic representation of large class indexed datatypes by 
    induction on the index type. Datatypes we can describe using this universe 
    include \ensuremath{\HSCon{Fin}} (\cref{lst:deffin}), \ensuremath{\HSCon{Vec}} (\cref{lst:defvec}) and closed 
    lambda terms (\cref{lst:defwellscoped}).

  \paragraph{Indexed Descriptions}
    Using the universe of \emph{Indexed Descriptions} \cite{dagand2013cosmology}
    we can represent arbitrary indexed datatypes. This allows us to describe 
    datatypes that are beyond what can be described using indexed containers, 
    that is, datatypes with recursive subtrees that are interdependent or whose 
    recursive subtrees have indices that cannot be uniquely determined from the 
    index of a value. 

\chapter{Literature Review}
In this section, we discuss some of the existing literature that is relevant in the domain of generating test data for property based testing. We take a look at some existing testing libraries, techniques for generation of constrained test data, and a few type universes beyond those we used that aim to describe (at least a subset of) indexed datatypes. 

\section{Property Based Testing}

  \textit{Property Based Testing} aims to assert properties that universally hold 
  for our programs by parameterizing tests over values and checking them against a 
  collection of test values. Libraries for property based testing often include some 
  kind of mechanism to automatically generate collections of test values. Existing 
  tools take different approaches towards generation of test data: \textit
  {QuickCheck} \cite{claessen2011quickcheck} randomly generates values within the 
  test domain, while \textit{SmallCheck} \cite{runciman2008smallcheck} and \textit
  {LeanCheck} \cite{matela2017tools} exhaustively enumerate all values in the test 
  domain up to a certain point. 

\subsection{Existing Libraries}

  There exist many libraries for property based testing. For brevity, we constrain ourselves here to those that are relevant in the domain of functional programming and/or haskell. 

\subsubsection{QuickCheck} 

  Published in 2000 by Claessen \& Hughes \cite{claessen2011quickcheck}, QuickCheck implements property based testing for Haskell. As mentioned before, test values are generated by sampling randomly from the domain of test values. QuickCheck supplies the typeclass \texttt{Arbitrary}, whose instances are those types for which random values can be generated. A property of type \ensuremath{\HSVar{a}\HSSym{\to} \HSCon{Bool}} can be tested if \ensuremath{\HSVar{a}} is an instance of \texttt{Arbitrary}. Instances for most common Haskell types are supplied by the library. If a property fails on a testcase, QuickCheck supplies a counterexample. Consider the following faulty definition of \ensuremath{\HSVar{reverse}}: 

\begin{myhaskell}
\begin{hscode}\SaveRestoreHook
\column{B}{@{}>{\hspre}l<{\hspost}@{}}%
\column{17}{@{}>{\hspre}c<{\hspost}@{}}%
\column{17E}{@{}l@{}}%
\column{20}{@{}>{\hspre}l<{\hspost}@{}}%
\column{E}{@{}>{\hspre}l<{\hspost}@{}}%
\>[B]{}\HSVar{reverse}\HSSym{::}\HSCon{Eq}\;\HSVar{a}\HSSym{\Rightarrow} \HSSpecial{\HSSym{[\mskip1.5mu} }\HSVar{a}\HSSpecial{\HSSym{\mskip1.5mu]}}\HSSym{\to} \HSSpecial{\HSSym{[\mskip1.5mu} }\HSVar{a}\HSSpecial{\HSSym{\mskip1.5mu]}}{}\<[E]%
\\
\>[B]{}\HSVar{reverse}\;\HSSpecial{\HSSym{[\mskip1.5mu} }\HSSpecial{\HSSym{\mskip1.5mu]}}{}\<[17]%
\>[17]{}\HSSym{\mathrel{=}}{}\<[17E]%
\>[20]{}\HSSpecial{\HSSym{[\mskip1.5mu} }\HSSpecial{\HSSym{\mskip1.5mu]}}{}\<[E]%
\\
\>[B]{}\HSVar{reverse}\;\HSSpecial{(}\HSVar{x}\HSCon{\mathbin{:}}\HSVar{xs}\HSSpecial{)}{}\<[17]%
\>[17]{}\HSSym{\mathrel{=}}{}\<[17E]%
\>[20]{}\HSVar{nub}\;\HSSpecial{(}\HSSpecial{(}\HSVar{reverse}\;\HSVar{xs}\HSSpecial{)}\HSSym{\plus} \HSSpecial{\HSSym{[\mskip1.5mu} }\HSVar{x}\HSSpecial{,}\HSVar{x}\HSSpecial{\HSSym{\mskip1.5mu]}}\HSSpecial{)}{}\<[E]%
\ColumnHook
\end{hscode}\resethooks
\end{myhaskell}

  If we now test our function by calling \ensuremath{\HSVar{quickCheck}\;\HSVar{reverse\char95 preserves\char95 length}}, we 
  get the following output: 

\begin{tabbing}\ttfamily
~Test\char46{}QuickCheck\char62{}~quickCheck~reverse\char95{}preserves\char95{}length~\\
\ttfamily ~\char42{}\char42{}\char42{}~Failed\char33{}~Falsifiable~\char40{}after~8~tests~and~2~shrinks\char41{}\char58{}~~~~\\
\ttfamily ~\char91{}7\char44{}7\char93{}
\end{tabbing}

  We see that a counterexample was found after 8 tests \textit{and 2 shrinks}. Due to 
  the random nature of the tested values, the counterexamples that falsify a property 
  are almost never minimal counterexamples. QuickCheck takes a counterexample and 
  applies some function that produces a collection of values that are smaller than the 
  original counterexample, and attempts to falsify the property using one of the 
  smaller values. By repeatedly \textit{Shrinking} a counterexample, QuickCheck is 
  able to find much smaller counterexamples, which are in general of much more use to 
  the programmer. 

  Perhaps somewhat surprising is that QuickCheck is also able randomly generate values 
  for function types by modifying the seed of the random generator (which is used to 
  generate the function's output) based on it's input. 

\subsubsection{(Lazy) SmallCheck} 

  Contrary to QuickCheck, SmallCheck \cite{runciman2008smallcheck} takes an \textit
  {enumerative} approach to the generation of test data. While the approach to 
  formulation and testing of properties is largely similar to QuickCheck's, test 
  values are not generated at random, but rather exhaustively enumerated up to a 
  certain \textit{depth}. Zero-arity constructors have depth $0$, while the depth of 
  any positive arity constructor is one greather than the maximum depth of its 
  arguments.  The motivation for this is the \textit{small scope hypothesis}: if a 
  program is incorrect, it will almost allways fail on some small input \cite
  {andoni2003evaluating}. 

  In addition to SmallCheck, there is also \textit{Lazy} SmallCheck. In many cases, 
  the value of a property is determined only by part of the input. Additionally, 
  Haskell's lazy semantics allow for functions to be defined on partial inputs. The 
  prime example of this is a property \texttt{sorted :: Ord a => [a] -> Bool} that 
  returns \texttt{false} when presented with \texttt{1:0:$\bot$}. It is not necessary 
  to evaluate $\bot$ to determine that the input list is not ordered. 

  Partial values represent an entire class of values. That is, \texttt{1:0:$\bot$} can 
  be viewed as a representation of the set of lists that have prefix \texttt{[1, 0]}. 
  By checking properties on partial values, it is possible to falsify a property for 
  an entire class of values in one go, in some cases greatly reducing the amount of 
  testcases needed. 

\subsubsection{LeanCheck} 

  Where SmallCheck uses a value's \textit{depth} to bound the number of test values, 
  LeanCheck uses a value's \textit{size} \cite{matela2017tools}, where size is defined 
  as the number of construction applications of positive arity.

  Both SmallCheck and LeanCheck contain functionality to enumerate functions similar 
  to QuickCheck's \texttt{Coarbitrary}. 

\subsubsection{Hegdgehog} 
  
  Hedgehog \cite{hedgehog} is a framework similar to QuickCheck, that aims to be a more modern alternative. It includes support for monadic effects in generators and concurrent checking of properties.

\subsubsection{Feat} 
  
  A downside to both SmallCheck and LeanCheck is that they do not 
  provide an efficient way to generate or sample large test values. QuickCheck has no 
  problem with either, but QuickCheck generators are often more tedious to write 
  compared to their SmallCheck counterpart. Feat \cite{duregaard2013feat} aims to fill 
  this gap by providing a way to efficiently enumerate algebraic types, employing 
  memoization techniques to efficiently find the $n^{th}$ element of an enumeration. 

\subsubsection{QuickChick} 
  
  QuickChick is a QuickCheck clone for the proof assistant Coq 
  \cite{denes2014quickchick}. The fact that Coq is a proof assistant enables the user 
  to reason about the testing framework itself \cite{paraskevopoulou2015foundational}. 
  This allows one, for example, to prove that generators adhere to some distribution.  

\subsection{Generating Constrained Test Data}\label{genconstrainedtd}

  Defining a suitable generation of test data for property based testing is 
  notoriously difficult in many cases, independent of whether we choose to sample from 
  or enumerate the space of test values. Writing generators for mutually recursive 
  datatypes with a suitable distribution is especially challenging. 
    
  We run into prolems when we desire to generate test data for properties with a 
  precondition. If a property's precondition is satisfied by few input values, it 
  becomes unpractical to test such a property by simply generating random input data. 
  Few testcases will be relevant (meaning they satisfy the precondition), and the 
  testcases that do are often trivial cases. The usual solution to this problem is to 
  define a custom test data generator that only produces data that satisfies the 
  precondition. In some cases, such as the \ensuremath{\HSVar{insert\char95 preserves\char95 sorted}} from section \ref
  {introduction}, a suitable generator is not too hard to define: 

\begin{myhaskell}
\begin{hscode}\SaveRestoreHook
\column{B}{@{}>{\hspre}l<{\hspost}@{}}%
\column{3}{@{}>{\hspre}l<{\hspost}@{}}%
\column{10}{@{}>{\hspre}l<{\hspost}@{}}%
\column{23}{@{}>{\hspre}l<{\hspost}@{}}%
\column{E}{@{}>{\hspre}l<{\hspost}@{}}%
\>[B]{}\HSVar{gen\char95 sorted}\HSSym{::}\HSCon{Gen}\;\HSSpecial{\HSSym{[\mskip1.5mu} }\HSCon{Int}\HSSpecial{\HSSym{\mskip1.5mu]}}{}\<[E]%
\\
\>[B]{}\HSVar{gen\char95 sorted}\HSSym{\mathrel{=}}\HSVar{arbitrary}\HSSym{\bind} \HSVar{return}\HSSym{\mathbin{\circ}}\HSVar{diff}{}\<[E]%
\\
\>[B]{}\hsindent{3}{}\<[3]%
\>[3]{}\HSKeyword{where}\;{}\<[10]%
\>[10]{}\HSVar{diff}\HSSym{::}\HSSpecial{\HSSym{[\mskip1.5mu} }\HSCon{Int}\HSSpecial{\HSSym{\mskip1.5mu]}}\HSSym{\to} \HSSpecial{\HSSym{[\mskip1.5mu} }\HSCon{Int}\HSSpecial{\HSSym{\mskip1.5mu]}}{}\<[E]%
\\
\>[10]{}\HSVar{diff}\;\HSSpecial{\HSSym{[\mskip1.5mu} }\HSSpecial{\HSSym{\mskip1.5mu]}}{}\<[23]%
\>[23]{}\HSSym{\mathrel{=}}\HSSpecial{\HSSym{[\mskip1.5mu} }\HSSpecial{\HSSym{\mskip1.5mu]}}{}\<[E]%
\\
\>[10]{}\HSVar{diff}\;\HSSpecial{(}\HSVar{x}\HSCon{\mathbin{:}}\HSVar{xs}\HSSpecial{)}{}\<[23]%
\>[23]{}\HSSym{\mathrel{=}}\HSVar{x}\HSCon{\mathbin{:}}\HSVar{map}\;\HSSpecial{(}\HSSym{+}\HSVar{x}\HSSpecial{)}\;\HSSpecial{(}\HSVar{diff}\;\HSVar{xs}\HSSpecial{)}{}\<[E]%
\ColumnHook
\end{hscode}\resethooks
\end{myhaskell}

  However, for more complex preconditions defining suitable generators is all but trivial. 

\subsection{Automatic Generation of Specifications}

  A surprising application of property based testing is the automatic generation of 
  program specifications, proposed by Claessen et al. \cite{claessen2010quickspec} 
  with the tool \textit{QuickSpec}. QuickSpec automatically generates a set of 
  candidate formal specifications given a list of pure functions, specifically in the 
  form of algebraic equations. Random property based testing is then used to falsify 
  specifications. In the end, the user is presented with a set of equations for which 
  no counterexample was found. 

\section{Techniques for Generating Test Data}

  This section discusses some existing work regarding the generation of test data 
  satisfying invariants, such as well-formed $\lambda$-terms. 

\subsection{Lambda Terms} 

  A problem often considered in literature is the generation of (well-typed) lambda 
  terms \cite{palka2011testing, grygiel2013counting, claessen2015generating}. Good 
  generation of arbitrary program terms is especially interesting in the context of 
  testing compiler infrastructure, and lambda terms provide a natural first step 
  towards that goal. 

  Claessen and Duregaard \cite{claessen2015generating} adapt the techniques described 
  by Duregaard \cite{duregaard2013feat} to allow efficient generation of constrained 
  data. They use a variation on rejection sampling, where the space of values is 
  gradually refined by rejecting classes of values through partial evaluation (similar 
  to SmallCheck \cite{runciman2008smallcheck}) until a value satisfying the imposed 
  constrained is found. 

  An alternative approach centered around the semantics of the simply typed lambda 
  calculus is described by Pa{\l}ka et al. \cite{palka2011testing}. Contrary to the 
  work done by Claessen and Duregaard \cite{claessen2015generating}, where 
  typechecking is viewed as a black box, they utilize definition of the typing rules 
  to devise an algorithm for generation of random lambda terms. The basic approach is 
  to take some input type, and randomly select an inference rule from the set of rules 
  that could have been applied to arrive at the goal type. Obviously, such a procedure 
  does not guarantee termination, as repeated application of the function application 
  rule will lead to an arbitrarily large goal type. As such, the algorithm requires a 
  maximum search depth and backtracking in order to guarantee that a suitable term 
  will eventually be generated, though it is not guaranteed that such a term exists if 
  a bound on term size is enforced \cite{moczurad2000statistical}. 

  Wang \cite{wang2005generating} considers the problem of generating closed untyped 
  lambda terms. 

\subsection{Inductive Relations in Coq}

  An approach to generation of constrained test data for Coq's QuickChick was proposed 
  by Lampropoulos et al. \cite{lampropoulos2017generating} in their 2017 paper \textit
  {Generating Good Generators for Inductive Relations}. They observe a common pattern 
  where the required test data is of a simple type, but constrained by some 
  precondition. The precondition is then given by some inductive dependent relation 
  indexed by said simple type. The \ensuremath{\HSCon{Sorted}} datatype shown in section \ref
  {introduction} is a good example of this

  They derive generators for such datatypes by abstracting over dependent inductive 
  relations indexed by simple types. For every constructor, the resulting type uses a 
  set of expressions as indices, that may depend on the constructor's arguments and 
  universally quantified variables. These expressions induce a set of unification 
  constraints that apply when using that particular constructor. These unification 
  constraints are then used when constructing generators to ensure that only values 
  for which the dependent inductive relation is inhabited are generated. 

\section{Generic Programming \& Type Universes}\label{sec:lituniverses}

  Datatype generic programming concerns techniques that allow for the definition of 
  functions by inducting on the \textit{structure} of a datatype. Many approaches 
  towards this goal have been developed, some more expressive than others. This 
  section discusses a few of them.  

\subsection{SOP (Sum of Products)}\label{sop}

  On of the more simple representations is the so called \textit{Sum of Products} view 
  \cite{de2014true}, where datatypes are respresented as a choice between an arbitrary 
  amount of constructors, each of which can have any arity. This view corresponds to 
  how datatypes are defined in Haskell. As we will see (for example in section \ref
  {patternfunctors}), other universes too employ sum and product combinators to 
  describe the structure of datatypes, though they do not necessarily enforce the 
  representation to be in disjunctive normal form. 

  Sum of Products, in its simplest form, cannot represent mutually recursive families 
  of datatypes. An extension that allows this has been developed in \cite
  {miraldo2018sums}. 

\chapter{A Combinator Library for Generators}
\section{The Type of Generators}

  We have not yet specified what it is exactly that we mean when we talk about \textit{generators}. In the context of property based testing, it makes sense to think of generators as entities that produce values of a certain type; the machinery that is responsible for supplying suitable test values. As we saw in section \cref{sec:literature}, this can mean different things depending on the library that you are using. \textit{SmallCheck} and \textit{LeanCheck} generators are functions that take a size parameter as input and produce an exhaustive list of all values that are smaller than the generator's input, while \textit{QuickCheck} generators randomly sample values of the desired type. Though various libraries use different terminology to refer to the mechanisms used to produce test values, we will use \textit{generator} as an umbrella term to refer to the test data producing parts of existing libraries. 

  \subsection{Examples in Existing Libraries}
  
   When comparing generator definitions across libraries, we see that their definition is often more determined by the structure of the datatype they ought to produce values of than the type of the generator itself. Let us consider the \ensuremath{\HSCon{Nat}} datatype (definition \ref{defnat}). In QuickCheck, we could define a generator for the \ensuremath{\HSCon{Nat}} datatype as follows: 

\begin{hscode}\SaveRestoreHook
\column{B}{@{}>{\hspre}l<{\hspost}@{}}%
\column{3}{@{}>{\hspre}l<{\hspost}@{}}%
\column{21}{@{}>{\hspre}l<{\hspost}@{}}%
\column{E}{@{}>{\hspre}l<{\hspost}@{}}%
\>[3]{}\HSVar{genNat}\HSSym{::}\HSCon{Gen}\;\HSCon{Nat}{}\<[E]%
\\
\>[3]{}\HSVar{genNat}\HSSym{\mathrel{=}}\HSVar{oneof}\;\HSSpecial{\HSSym{[\mskip1.5mu} }{}\<[21]%
\>[21]{}\HSVar{pure}\;\HSCon{Zero}\HSSpecial{,}\HSCon{Suc}\HSSym{<\$>}\HSVar{genNat}\HSSpecial{\HSSym{\mskip1.5mu]}}{}\<[E]%
\ColumnHook
\end{hscode}\resethooks

  QuickCheck includes many combinators to finetune the distribution of values of the generated type, which are omitted in this case since they do not structurally alter the generator. Compare the above generator to its SmallCheck equivalent: 

\begin{hscode}\SaveRestoreHook
\column{B}{@{}>{\hspre}l<{\hspost}@{}}%
\column{3}{@{}>{\hspre}l<{\hspost}@{}}%
\column{24}{@{}>{\hspre}c<{\hspost}@{}}%
\column{24E}{@{}l@{}}%
\column{28}{@{}>{\hspre}l<{\hspost}@{}}%
\column{E}{@{}>{\hspre}l<{\hspost}@{}}%
\>[B]{}\HSKeyword{instance}\;\HSCon{Serial}\;\HSVar{m}\;\HSCon{Nat}\;\HSKeyword{where}{}\<[E]%
\\
\>[B]{}\hsindent{3}{}\<[3]%
\>[3]{}\HSVar{series}\HSSym{\mathrel{=}}\HSVar{cons0}\;\HSCon{Zero}{}\<[24]%
\>[24]{}\HSSym{\char92 /}{}\<[24E]%
\>[28]{}\HSCon{Cons1}\;\HSCon{Suc}{}\<[E]%
\ColumnHook
\end{hscode}\resethooks

  Both generator definitions have a strikingly similar structure, marking a choice between the two available constructors (\ensuremath{\HSCon{Zero}} and \ensuremath{\HSCon{Suc}}) and employing a appropriate combinators to produce values for said constructors. Despite this structural similarity, the underlying types of the respective generators are wildly different, with \ensuremath{\HSVar{genNat}} being an \ensuremath{\HSCon{IO}} operation that samples random values and the \ensuremath{\HSCon{Serial}} instance being a function taking a depth and producing all values up to that depth. 

\subsection{Separating Structure and Interpretation}

  The previous example suggests that there is a case to be made for separating a generators structure from the format in which test values are presented. Additionally, by having a single datatype representing a generator's structure, we shift the burden of proving termination from a generator's definition to its interpretation, which in Agda is a considerable advantage. In practice this means that we define some datatype \ensuremath{\HSCon{Gen}\;\HSVar{a}} that marks the structure of a generator, and a function \ensuremath{\HSVar{interpret}\HSCon{\mathbin{:}}\HSCon{Gen}\;\HSVar{a}\HSSym{\to} \HSCon{T}\;\HSVar{a}} that maps an input structure to some \ensuremath{\HSCon{T}\;\HSVar{a}}, where \ensuremath{\HSCon{T}} which actually produces test values. In our case, we will almost exclusively consider an interpretation of generators to functions of type \ensuremath{\HSCon{ℕ}\HSSym{→}\HSCon{List}\;\HSVar{a}}, but we could have chosen \ensuremath{\HSCon{T}} to by any other type of collection of values of type \ensuremath{\HSVar{a}}. An implication of this separation is that, given suitable interpretation functions, a user only has to define a single generator in order to be able to employ different strategies for generating test values, potentially allowing for both random and enumerative testing to be combined into a single framework. 

  This approach means that generator combinators are not functions that operate on a a generator's result, such as merging two streams of values, but rather a constructor of some abstract generator type; \ensuremath{\HSCon{Gen}} in our case. This datatype represents generators in a tree-like structure, not unlike the more familiar abstract syntax trees used to represent parsed programs. 

\subsection{The \ensuremath{\HSCon{Gen}} Datatype}

  We define the datatype of generators, \ensuremath{\HSCon{Gen}\;\HSVar{a}\;\HSVar{t}}, to be a family of types indexed by two types \footnote{The listed definition will not be accepted by Agda due to inconsistencies in the universe levels. This is also the case for many code examples to come. To keep things readable, we will not concern ourselves with universe levels throughout this thesis.}. One signifying the type of values that are produced by the generator, and one specifying the type of values produced by recursive positions. 

\includeagdalisting{4}{gendef}{Definition of the \ensuremath{\HSCon{Gen}} datatype}{lst:gendef}

  \textit{Closed} generators are then generators produce that produce the values of the same type as their recursive positions: 

\includeagda{4}{gdef}

  The \ensuremath{\HSCon{Pure}} and \ensuremath{\HSCon{Ap}} constructors make \ensuremath{\HSCon{Gen}} an instance of \ensuremath{\HSCon{Applicative}}, meaning that we can (given a fancy operator for denoting choice) denote generators in way that is very similar to their definition: 

\includeagda{4}{gennat}

  This serves to emphasize that the structure of generators can, in the case of simpler datatypes, be mechanically derived from the structure of a datatype. We will see how this can be done in chapter \cref{chap:derivingregular}. 

  The question remains how to deal with constructors that refer to \textit{other} types. For example, consider the type of lists (definition \ref{deflist}). We can define an appropriate generator following the structure of the datatype definition: 

\includeagda{4}{listgenhole}

  It is however not immediately clear what value to supply to the remaining interaction point. If we inspect its goal type we see that we should supply a value of type \ensuremath{\HSCon{Gen}\;\HSVar{a}\;\HSSpecial{(}\HSCon{List}\;\HSVar{a}\HSSpecial{)}}: a generator producing values of type \ensuremath{\HSVar{a}}, with recursive positions producing values of type \ensuremath{\HSCon{List}\;\HSVar{a}}. This makes little sense, as we would rather be able to invoke other \textit{closed generators} from within a generator. To do so, we add another constructor to the \ensuremath{\HSCon{Gen}} datatype, that signifies the invokation of a closed generator for another datatype: 

\includeagda{4}{calldef}

  Using this definition of \ensuremath{\HSCon{Call}}, we can complete the previous definition for \ensuremath{\HSVar{list}}: 

\includeagda{4}{listgen}

\subsection{Generator Interpretations}

  We can view a generator's interpretation as any function mapping generators to some type, where the output type is parameterized by the type of values produced by a generator: 

\includeagda{4}{intdef}

  From this definition of \ensuremath{\HSCon{Interpretation}}, we can define concrete interpretations. For example, if we want to behave our generators similar to SmallCheck's \ensuremath{\HSCon{Series}}, we might define the following concrete instantiation of the \ensuremath{\HSCon{Interpretation}} type: 

\includeagda{4}{scdef}

  We can then define a generator's behiour by supplying a definition that inhabits the \ensuremath{\HSCon{GenAsList}} type: 

\includeagda{4}{scint}

  The goal type of the open interaction point is then $\mathbb{N}$\ensuremath{\HSSym{→}\HSCon{List}\;\HSVar{a}}. We will see in \cref{sec:enuminterpretation} how we can flesh out this particular interpretation. We could however have chosen any other result type, depending on what suits our particular needs. An alternative would be to interpret generators as a \ensuremath{\HSCon{Colist}}, omitting the depth bound altogether:

\includeagda{4}{intcolist}

\section{Generalization to Indexed Datatypes}

  A first approximation towards a generalization of the \ensuremath{\HSCon{Gen}} type to indexed types might be to simply lift the existing definition from \ensuremath{\HSCon{Set}} to \ensuremath{\HSCon{I}\HSSym{→}\HSCon{Set}}. 

\includeagda{4}{liftgen}

  However, by doing so we implicitly impose the constraint that the recursive positions of a value have the same index as the recursive positions within it. Consider, for example, the \ensuremath{\HSCon{Fin}} type (definition \ref{findef}). If we attempt to define a generator using the lifted type, we run into a problem. 

\includeagda{4}{finhole}

  Any attempt to fill the open interaction point with the \ensuremath{\HSVar{μ}} constructor fails, as it expects a value of \ensuremath{\HSCon{Gen}\;\HSSpecial{(}\HSCon{Fin}\;\HSVar{n}\HSSpecial{)}\;\HSSpecial{(}\HSCon{Fin}\;\HSVar{suc}\;\HSVar{n}\HSSpecial{)}}, but \ensuremath{\HSVar{μ}} requires both its type parameters to be equal. We can circumvent this issue by using direct recursion. 

\includeagda{4}{findirect}

  It is however clear that this approach becomes a problem once we attempt to define generators for datatypes with recursive positions which have indices that are not structurally smaller than the index they target. To overcome these limitations we resolve to a separate deep embedding of generators for indexed types. 

\includeagdalisting{4}{genidef}{Definitiong of the \ensuremath{\HSCon{Genᵢ}} datatype}{lst:genidef}

  And consequently the type of closed indexed generators. 

\includeagda{4}{gidef}

  Notice how the \ensuremath{\HSCon{Apᵢ}} constructor allows for its second argument to have a different index. The reason for this becomes clear when we 

  With the same combinators as used for the \ensuremath{\HSCon{Gen}} type, we can now define a generator for the \ensuremath{\HSCon{Fin}} type. 

\includeagda{4}{genfin}

  Now defining generators for datatypes with recursive positions whose indices are not structurally smaller than the index of the datatype itself can be done without complaints from the termination checker, such as well-scoped $\lambda$-terms (definition \ref{defwellscoped}). 

\includeagda{4}{wellscoped}

  It is important to note that it is not possible to call indexed generators from simple generators and vice versa with this setup. We can allow this by either parameterizing the \ensuremath{\HSCon{Call}} and \ensuremath{\HSVar{iCall}} constructors with the datatype they refer to, or by adding extra constructors to the \ensuremath{\HSCon{Gen}} and \ensuremath{\HSCon{Genᵢ}} datatypes, making them mutually recursive. 

\section{Interpreting Generators as Enumerations}\label{sec:enuminterpretation}

  We will now consider an example interpretation of generators where we map values of the \ensuremath{\HSCon{Gen}} or \ensuremath{\HSCon{Genᵢ}} datatypes to functions of type $\mathbb{N}$\ensuremath{\HSSym{→}\HSCon{List}\;\HSVar{a}}. The constructors of both datatypes mimic the combinators used Haskell's \ensuremath{\HSCon{Applicative}} and \ensuremath{\HSCon{Alternative}} typeclasses, so we can use the \ensuremath{\HSCon{List}} instances of these typeclasses for guidance when defining an enumerative interpretation.  

\includeagdalisting{4}{tolist}{Interpretation of the \ensuremath{\HSCon{Gen}} datatype as an enumeration}{lst:tolist}

  Similarly, we can define such an interpretation for the \ensuremath{\HSCon{Genᵢ}} datatype similar to listing \ref{lst:tolist} with the only difference being the appropriate indices getting passed to recursive calls. Notice how our generator's behaviour - most notably the intended semantics of the input depth bound - is entirely encoded within the definition of the interpretation. In this case by decrementing \ensuremath{\HSVar{n}} anytime a recursive position is encountered.  

\section{Properties for Enumerations}

\section{Generating Function Types}

\section{Monadic Generators}

  There are some cases in which the applicative combinators are not expressive enough to capture the desired generator. For example, if we were to define a construction for generation of $\Sigma$ types, we encounter some problems. 

\includeagda{4}{sigmagenhole}

  We can extend the \ensuremath{\HSCon{Gen}} datatype with a \ensuremath{\HSCon{Bind}} operation that mimics the monadic bind operator (\ensuremath{\HSSym{\bind} }) to allow for such dependencies to exist between generated values.

\includeagda{4}{sigmagen}

\chapter{Generic Generators for Regular types}

  A large class of recursive algebraic data types can be described with the universe 
  of \emph{regular types}. In this section we lay out this universe, together with its 
  semantics, and describe how we may define functions over regular types by induction 
  over their codes. We will then show how this allows us to derive from a code a 
  generic generator that produces all values of a regular type. We sketch how we can 
  prove that these generators are indeed complete. 

\section{The universe of regular types}

  Though the exact definition may vary across sources, the universe of regular types 
  is generally regarded to consist of the \emph{empty type} (or $\mathbb{0}$), the 
  unit type (or $\mathbb{1}$) and constants types. It is closed under both products 
  and coproducts \footnote{This roughly corresponds to datatypes in Haskell 98}. We 
  can define a datatype for this universe in Agda as shown in lising \ref{lst:regular}

\includeagdalisting{5}{regular}{The universe of regular types}{lst:regular}

  The semantics associated with the \ensuremath{\HSCon{Reg}} datatype, as shown in listing \ref
  {lst:regsem}, map a code to a functorial representation of a datatype, commonly 
  known as its \emph{pattern functor}. The datatype that is represented by a code is 
  isomorphic to the least fixpoint of its pattern functor. 

\includeagdalisting{5}{regularsemantics}{Semantics of the universe of regular types}{lst:regsem}

  \begin{example}

    The type of natural numbers (see listing \ref{lst:defnat}) 
    exposes two constructors: the nullary constructor \ensuremath{\HSVar{zero}}, and the unary constructor \ensuremath{\HSVar{suc}} that takes one recursive argument. We may thus view this type as a coproduct (i.e. choice) of either a \emph{unit type} or a \emph{recursive subtree}: 

\includeagdanv{5}{natregular}

    We convince ourselves that \ensuremath{\HSCon{ℕ'}} is indeed equivalent to \ensuremath{\HSCon{ℕ}} by defining conversion 
    functions, and showing their composition is extensionally equal to the identity 
    function, shown in listing \ref{lst:natiso}. 

  \end{example}

\includeagdalisting{5}{natiso}{Isomorphism between \ensuremath{\HSCon{ℕ}} and \ensuremath{\HSCon{ℕ'}}}{lst:natiso}

  We may then say that a type is regular if we can provide a proof that it is 
  isomorphic to the fixpoint of some \ensuremath{\HSVar{c}} of type \ensuremath{\HSCon{Reg}}. We use a record to capture 
  this notion, consisting of a code and an value that witnesses the isomorphism.

\includeagda{5}{regularrecord}

  By instantiating \ensuremath{\HSCon{Regular}} for a type, we may use any generic functionality that is defined over regular types. 

\subsection{Non-regular data types}

  Although there are many algebraic datatypes that can be described in the universe 
  of regular types, some cannot. Perhaps the most obvious limitation the is lack of 
  ability to caputure data families indexed with values. The regular univeres 
  imposes the implicit restriction that a datatype is uniform in the sens that all 
  recursive subtrees are of the same type. Indexed families, however, allow for 
  recursive subtrees to have a structure that is different from the structure of the 
  datatype they are a part of. 

  Furethermore, any family of mutually recursive datatypes cannot be described as a 
  regular type; again, this is a result of the restriction that recursive positions 
  allways refer to a datatype with the same structure. 

\section{Generic Generators for regular types}

  We can derive generators for all regular types by induction over their associated 
  codes. Furthermore, we will show in section \cref{regularproof} that, once 
  interpreted as enumerators, these generators are complete; i.e. any value will eventually show up in the enumerator, provided we supply a sufficiently large size parameter.  

\subsection{Defining functions over codes}

  If we apply the approach described in \cref{sec:tudesignpattern} without care, we run into problems. Simply put, we cannot work with values of type \ensuremath{\HSCon{Fix}\;\HSVar{c}}, since this implicitly imposes the restriction that any \ensuremath{\HSCon{I}} in \ensuremath{\HSVar{c}} refers to \ensuremath{\HSCon{Fix}\;\HSVar{c}}. However, as we descent into recursive calls, the code we are working with changes, and with it the type associated with recursive positions. For example: the \ensuremath{\HSCon{I}} in (\ensuremath{\HSCon{U}\HSSym{⊕}\HSCon{I}}) refers to values of type \ensuremath{\HSCon{Fix}\;\HSSpecial{(}\HSCon{U}\HSSym{⊕}\HSCon{I}\HSSpecial{)}}, not \ensuremath{\HSCon{Fix}\;\HSCon{I}}. We need to make a distinction between the code we are currently working on, and the code that recursive positions refer to. For this reason, we cannot define the generic generator, \ensuremath{\HSVar{deriveGen}}, with the following type signature: 

\includeagda{5}{genericgen}

  If we observe that \ensuremath{\HSSym{⟦}\HSVar{c}\HSSym{⟧}\HSSpecial{(}\HSCon{Fix}\;\HSVar{c}\HSSpecial{)}\HSSym{≃}\HSCon{Fix}\;\HSVar{c}}, we may alter the type signature of \ensuremath{\HSVar{deriveGen}} slightly, such that it takes two input codes instead of one

\includeagda{5}{genericgen2}

  This allows us to induct over the first input code, while still being able to have recursive positions reference the correct \emph{top-level code}. Notice that the first and second type parameter of \ensuremath{\HSCon{Gen}} are different. This is intensional, as we would otherwise not be able to use the $\mu$ constructor to mark recursive positions.  

\subsection{Composing generic generators}

  Now that we have the correct type for \ensuremath{\HSVar{deriveGen}} in place, we can start defining it. Starting with the cases for \ensuremath{\HSCon{Z}} and \ensuremath{\HSCon{U}}: 

\includeagda{5}{genericgenZU}

  Both cases are trivial. In case of the \ensuremath{\HSCon{Z}} combinator, we yield a generator that produces no elements. As for the \ensuremath{\HSCon{U}} combinator, \ensuremath{\HSSym{⟦}\HSCon{U}\HSSym{⟧}\HSSpecial{(}\HSCon{Fix}\;\HSVar{c'}\HSSpecial{)}} equals \ensuremath{\HSSym{⊤}}, so we need to return a generator that produces all inhabitants of \ensuremath{\HSSym{⊤}}. This is simply done by lifting the single value \ensuremath{\HSVar{tt}} into the generator type. 

  In case of the \ensuremath{\HSCon{I}} combinator, we cannot simply use the $\mu$ constructor right away. In this context, $\mu$ has the type \ensuremath{\HSCon{Gen}\;\HSSpecial{(}\HSSym{⟦}\HSVar{c'}\HSSym{⟧}\HSSpecial{(}\HSCon{Fix}\;\HSVar{c'}\HSSpecial{)}\HSSpecial{)}\;\HSSpecial{(}\HSSym{⟦}\HSVar{c'}\HSSym{⟧}\HSSpecial{(}\HSCon{Fix}\;\HSVar{c'}\HSSpecial{)}\HSSpecial{)}}. However, since \ensuremath{\HSSym{⟦}\HSCon{I}\HSSym{⟧}\HSSpecial{(}\HSCon{Fix}\;\HSVar{c}\HSSpecial{)}} equals \ensuremath{\HSCon{Fix}\;\HSVar{c}}, the types do not lign up. We need to map the \ensuremath{\HSCon{In}} constructor over $\mu$ to fix this: 

\includeagda{5}{genericgenI}

  Moving on to products and coproducts: with the correct type for \ensuremath{\HSVar{deriveGen}} in place, we can define their generators quite easily by recursing on the left and right subcodes, and combining their results using the appropriate generator combinators: 

\includeagda{5}{genericgenPCOP}

  Since the the \ensuremath{\HSCon{Regular}} record expects an isomorphism with \ensuremath{\HSCon{Fix}\;\HSVar{c}}, we still need to wrap the resulting generator in the \ensuremath{\HSCon{In}} constructor: 

\includeagda{5}{genericgenFinal}

  The elements produced by \ensuremath{\HSVar{genericGen}} can now readily be transformed into the required datatype through an appropriate isomorphism. 

  \begin{example}

    We derive a generator for natural numbers by invoking \ensuremath{\HSVar{genericGen}} on the appropriate code \ensuremath{\HSCon{U}\HSSym{⊕}\HSCon{I}}, and applying the isomorphism defined in listing \ref{natiso} to its results: 

\includeagdanv{5}{genericgenNat}

  \end{example}

  In general, we can derive a generator for any type \ensuremath{\HSCon{A}}, as long as there is an instance argument of the type \ensuremath{\HSCon{Regular}\;\HSCon{A}} in scope: 

\includeagda{5}{isogen}

\section{Enumerators for regular types}

  We can define enumerators for regular types, and prove that these enumerators are \emph{complete}, in the sense that they will allways \emph{eventually} produce any arbitrary elementof the type they range over. We show how to arrive at such an enumerator, and how to assemble the completeness proof. 

\subsection{A generic enumerator}

  Since we define enumerators as functions \ensuremath{\HSCon{ℕ}\HSSym{→}\HSCon{List}\;\HSCon{A}}, we can look at Haskell's standard library for inspiration, specifically the implementations of the \ensuremath{\HSCon{Applicative}} and \ensuremath{\HSCon{Alternative}} typeclasses. 

\subsection{Proving completeness}\label{sec:regularproof}

\chapter{Deriving Generators for Indexed Containers}

\chapter{Deriving Generators for Indexed Descriptions}
\section{Universe Description}

  We utilize the generic description for indexed datatypes proposed by Dagand \cite{dagand2013cosmology} in his PhD thesis.

\subsection{Definition}

  Indexed descriptions are not much unlike the codes used to describe regular types (that is, the \ensuremath{\HSCon{Reg}} datatype), with the differences being: 

\begin{enumerate}
  \item 
  A type parameter \ensuremath{\HSCon{I}\HSCon{\mathbin{:}}\HSCon{Set}}, describing the type of indices.

  \item 
  A generalized coproduct, \ensuremath{\HSSpecial{`}}$\sigma$, that denotes choice between $n$ constructors, in favor of the \ensuremath{\HSSym{⊕}} combinator. 

  \item 
  Recursive positions storing the index of recursive values

  \item 
  Addition of a combinator to encode $\Sigma$ types which is a generalization of the \ensuremath{\HSCon{K}} combinator. 
\end{enumerate}

  This amounts to the definition of indexed descriptions described in listing \ref{lst:idesc}. 

\includeagdalisting{7}{idesc}{The Universe of indexed descriptions}{lst:idesc}\

The \ensuremath{\HSCon{Sl}} datatype is used to select the right branch from the generic coproduct, and is isomorphic to the \ensuremath{\HSCon{Fin}} datatype. 

\includeagda{7}{sl}

\subsection{Examples}

\section{Generic Generators for Indexed Descriptions}

\chapter{Program Term Generation}

\chapter{Implementation in Haskell}

\chapter{Conclusion \& Further Work}

\appendix
\chapter{Datatype Definitions}

\section{Natural numbers}

\begin{listing}{Definition of natural numbers in Haskell and Agda}{lst:defnat}

  \begin{hscode}\SaveRestoreHook
\column{B}{@{}>{\hspre}l<{\hspost}@{}}%
\column{5}{@{}>{\hspre}l<{\hspost}@{}}%
\column{15}{@{}>{\hspre}c<{\hspost}@{}}%
\column{15E}{@{}l@{}}%
\column{18}{@{}>{\hspre}l<{\hspost}@{}}%
\column{E}{@{}>{\hspre}l<{\hspost}@{}}%
\>[5]{}\HSKeyword{data}\;\HSCon{Nat}{}\<[15]%
\>[15]{}\HSSym{\mathrel{=}}{}\<[15E]%
\>[18]{}\HSCon{Zero}{}\<[E]%
\\
\>[15]{}\HSSym{\mid} {}\<[15E]%
\>[18]{}\HSCon{Suc}\;\HSCon{N}{}\<[E]%
\ColumnHook
\end{hscode}\resethooks

  \dotfill

  \appincludeagda{A}{nat}

\end{listing}

\section{Finite Sets}

\begin{listing}{Definition of finite sets in Agda}{lst:deffin}

  \appincludeagda{A}{fin}

\end{listing}
\newpage
\section{Vectors}

\begin{listing}{Definition of vectors (size-indexed listst) in Agda}{lst:defvec}

  \appincludeagda{A}{vec}

\end{listing}

\section{Simple Types}

\begin{listing}{Definition of simple types in Haskell and Agda}{lst:defstype}

  \begin{hscode}\SaveRestoreHook
\column{B}{@{}>{\hspre}l<{\hspost}@{}}%
\column{3}{@{}>{\hspre}l<{\hspost}@{}}%
\column{14}{@{}>{\hspre}c<{\hspost}@{}}%
\column{14E}{@{}l@{}}%
\column{17}{@{}>{\hspre}l<{\hspost}@{}}%
\column{E}{@{}>{\hspre}l<{\hspost}@{}}%
\>[3]{}\HSKeyword{data}\;\HSCon{Type}{}\<[14]%
\>[14]{}\HSSym{\mathrel{=}}{}\<[14E]%
\>[17]{}\HSCon{T}{}\<[E]%
\\
\>[14]{}\HSSym{\mid} {}\<[14E]%
\>[17]{}\HSCon{Type}\HSSym{:->:}\HSCon{Type}{}\<[E]%
\ColumnHook
\end{hscode}\resethooks

  \dotfill

  \appincludeagda{A}{simpletypes}

\end{listing}

\section{Contexts}

\begin{listing}{Definition of contexts in Haskell and Agda}{lst:defcontext}

  \begin{hscode}\SaveRestoreHook
\column{B}{@{}>{\hspre}l<{\hspost}@{}}%
\column{3}{@{}>{\hspre}l<{\hspost}@{}}%
\column{13}{@{}>{\hspre}c<{\hspost}@{}}%
\column{13E}{@{}l@{}}%
\column{16}{@{}>{\hspre}l<{\hspost}@{}}%
\column{E}{@{}>{\hspre}l<{\hspost}@{}}%
\>[3]{}\HSKeyword{data}\;\HSCon{Ctx}{}\<[13]%
\>[13]{}\HSSym{\mathrel{=}}{}\<[13E]%
\>[16]{}\HSCon{Empty}{}\<[E]%
\\
\>[13]{}\HSSym{\mid} {}\<[13E]%
\>[16]{}\HSCon{Cons}\;\HSCon{Id}\;\HSCon{Type}\;\HSCon{Ctx}{}\<[E]%
\ColumnHook
\end{hscode}\resethooks

  \dotfill

  \appincludeagda{A}{context}

\end{listing}
\newpage
\section{Raw $\lambda$-Terms}

\begin{listing}{Definition of raw $\lambda$-terms in Haskell and Agda}{lst:defrawterm}

  \begin{hscode}\SaveRestoreHook
\column{B}{@{}>{\hspre}l<{\hspost}@{}}%
\column{3}{@{}>{\hspre}l<{\hspost}@{}}%
\column{12}{@{}>{\hspre}c<{\hspost}@{}}%
\column{12E}{@{}l@{}}%
\column{15}{@{}>{\hspre}l<{\hspost}@{}}%
\column{E}{@{}>{\hspre}l<{\hspost}@{}}%
\>[3]{}\HSKeyword{data}\;\HSCon{RT}{}\<[12]%
\>[12]{}\HSSym{\mathrel{=}}{}\<[12E]%
\>[15]{}\HSCon{Var}\;\HSCon{Id}{}\<[E]%
\\
\>[12]{}\HSSym{\mid} {}\<[12E]%
\>[15]{}\HSCon{Abs}\;\HSCon{Id}\;\HSCon{RT}{}\<[E]%
\\
\>[12]{}\HSSym{\mid} {}\<[12E]%
\>[15]{}\HSCon{App}\;\HSCon{RT}\;\HSCon{RT}{}\<[E]%
\ColumnHook
\end{hscode}\resethooks

  \dotfill

  \appincludeagda{A}{rawterm}

\end{listing}

\section{Lists}

\begin{listing}{Definition lists and Agda}{lst:deflist}

  \appincludeagda{A}{list}

\end{listing}

\section{Well-scoped $\lambda$-terms}

\begin{listing}{Definition well-scoped $\lambda$-terms in Agda}{lst:defwellscoped}

  \appincludeagda{A}{wellscoped}

\end{listing}

\backmatter
\listoffigures
\listoftables

\bibliographystyle{alpha}
\bibliography{references}

\end{document}


