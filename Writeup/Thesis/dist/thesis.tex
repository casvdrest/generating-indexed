\documentclass[a4paper,msc,twosized=semi]{uustthesis}

\usepackage{framed}
\usepackage{mdframed}
\usepackage{setspace}
% \usepackage{extsizes}

\renewcommand{\figurename}{Listing}
\renewcommand{\listfigurename}{Code listings}

%% Listings 
\newenvironment{listing}[2] %% #1 = caption #2 = label
{
    \begin{figure}[h]
      \label{#2}
      \begin{framed}
        \caption{#1}
}
{
      \end{framed}
    \end{figure}
}

%% Agda snippets 
\newcommand{\includeagda}[2]{\vspace*{-0.35cm}\begin{center}\ExecuteMetaData[../src/chap0#1/latex/code.tex]{#2}\end{center}\vspace*{-0.35cm}}

%% Agda snippets, without removed spacing
\newcommand{\includeagdanv}[2]{\begin{center}\ExecuteMetaData[../src/chap0#1/latex/code.tex]{#2}\end{center}}

%% Agda snippets, not centered
\newcommand{\includeagdanc}[2]{\ExecuteMetaData[../src/chap0#1/latex/code.tex]{#2}\vspace*{-0.35cm}}

%% Agda listings
\newcommand{\includeagdalisting}[4]{
  \begin{listing}{#3}{#4} 
    \includeagdanc{#1}{#2}
  \end{listing} 
}

%% Agda snippets (appendices)
\newcommand{\appincludeagda}[2]{\ExecuteMetaData[../src/app#1/latex/code.tex]{#2}}

%% Agda listings (appendices)
\newcommand{\appincludeagdalisting}[4]{
  \begin{listing}{#3}{#4} 
    \appincludeagda{#1}{#2}
  \end{listing}
}

\newmdenv[
  topline=false,
  bottomline=false,
  rightline=false,
  skipabove=\topsep,
  skipbelow=\topsep
]{siderules}

\newenvironment{example}[0] 
{
  \begin{siderules}
    \vspace{-0.5cm}
    \paragraph{\textbf{Example}}
}
{
  \end{siderules}
}

%% ODER: format ==         = "\mathrel{==}"
%% ODER: format /=         = "\neq "
%
%
\makeatletter
\@ifundefined{lhs2tex.lhs2tex.sty.read}%
  {\@namedef{lhs2tex.lhs2tex.sty.read}{}%
   \newcommand\SkipToFmtEnd{}%
   \newcommand\EndFmtInput{}%
   \long\def\SkipToFmtEnd#1\EndFmtInput{}%
  }\SkipToFmtEnd

\newcommand\ReadOnlyOnce[1]{\@ifundefined{#1}{\@namedef{#1}{}}\SkipToFmtEnd}
\usepackage{amstext}
\usepackage{amssymb}
\usepackage{stmaryrd}
\DeclareFontFamily{OT1}{cmtex}{}
\DeclareFontShape{OT1}{cmtex}{m}{n}
  {<5><6><7><8>cmtex8
   <9>cmtex9
   <10><10.95><12><14.4><17.28><20.74><24.88>cmtex10}{}
\DeclareFontShape{OT1}{cmtex}{m}{it}
  {<-> ssub * cmtt/m/it}{}
\newcommand{\texfamily}{\fontfamily{cmtex}\selectfont}
\DeclareFontShape{OT1}{cmtt}{bx}{n}
  {<5><6><7><8>cmtt8
   <9>cmbtt9
   <10><10.95><12><14.4><17.28><20.74><24.88>cmbtt10}{}
\DeclareFontShape{OT1}{cmtex}{bx}{n}
  {<-> ssub * cmtt/bx/n}{}
\newcommand{\tex}[1]{\text{\texfamily#1}}	% NEU

\newcommand{\Sp}{\hskip.33334em\relax}


\newcommand{\Conid}[1]{\mathit{#1}}
\newcommand{\Varid}[1]{\mathit{#1}}
\newcommand{\anonymous}{\kern0.06em \vbox{\hrule\@width.5em}}
\newcommand{\plus}{\mathbin{+\!\!\!+}}
\newcommand{\bind}{\mathbin{>\!\!\!>\mkern-6.7mu=}}
\newcommand{\rbind}{\mathbin{=\mkern-6.7mu<\!\!\!<}}% suggested by Neil Mitchell
\newcommand{\sequ}{\mathbin{>\!\!\!>}}
\renewcommand{\leq}{\leqslant}
\renewcommand{\geq}{\geqslant}
\usepackage{polytable}

%mathindent has to be defined
\@ifundefined{mathindent}%
  {\newdimen\mathindent\mathindent\leftmargini}%
  {}%

\def\resethooks{%
  \global\let\SaveRestoreHook\empty
  \global\let\ColumnHook\empty}
\newcommand*{\savecolumns}[1][default]%
  {\g@addto@macro\SaveRestoreHook{\savecolumns[#1]}}
\newcommand*{\restorecolumns}[1][default]%
  {\g@addto@macro\SaveRestoreHook{\restorecolumns[#1]}}
\newcommand*{\aligncolumn}[2]%
  {\g@addto@macro\ColumnHook{\column{#1}{#2}}}

\resethooks

\newcommand{\onelinecommentchars}{\quad-{}- }
\newcommand{\commentbeginchars}{\enskip\{-}
\newcommand{\commentendchars}{-\}\enskip}

\newcommand{\visiblecomments}{%
  \let\onelinecomment=\onelinecommentchars
  \let\commentbegin=\commentbeginchars
  \let\commentend=\commentendchars}

\newcommand{\invisiblecomments}{%
  \let\onelinecomment=\empty
  \let\commentbegin=\empty
  \let\commentend=\empty}

\visiblecomments

\newlength{\blanklineskip}
\setlength{\blanklineskip}{0.66084ex}

\newcommand{\hsindent}[1]{\quad}% default is fixed indentation
\let\hspre\empty
\let\hspost\empty
\newcommand{\NB}{\textbf{NB}}
\newcommand{\Todo}[1]{$\langle$\textbf{To do:}~#1$\rangle$}

\EndFmtInput
\makeatother
%
%
%
%
%
%
% This package provides two environments suitable to take the place
% of hscode, called "plainhscode" and "arrayhscode". 
%
% The plain environment surrounds each code block by vertical space,
% and it uses \abovedisplayskip and \belowdisplayskip to get spacing
% similar to formulas. Note that if these dimensions are changed,
% the spacing around displayed math formulas changes as well.
% All code is indented using \leftskip.
%
% Changed 19.08.2004 to reflect changes in colorcode. Should work with
% CodeGroup.sty.
%
\ReadOnlyOnce{polycode.fmt}%
\makeatletter

\newcommand{\hsnewpar}[1]%
  {{\parskip=0pt\parindent=0pt\par\vskip #1\noindent}}

% can be used, for instance, to redefine the code size, by setting the
% command to \small or something alike
\newcommand{\hscodestyle}{}

% The command \sethscode can be used to switch the code formatting
% behaviour by mapping the hscode environment in the subst directive
% to a new LaTeX environment.

\newcommand{\sethscode}[1]%
  {\expandafter\let\expandafter\hscode\csname #1\endcsname
   \expandafter\let\expandafter\endhscode\csname end#1\endcsname}

% "compatibility" mode restores the non-polycode.fmt layout.

\newenvironment{compathscode}%
  {\par\noindent
   \advance\leftskip\mathindent
   \hscodestyle
   \let\\=\@normalcr
   \let\hspre\(\let\hspost\)%
   \pboxed}%
  {\endpboxed\)%
   \par\noindent
   \ignorespacesafterend}

\newcommand{\compaths}{\sethscode{compathscode}}

% "plain" mode is the proposed default.
% It should now work with \centering.
% This required some changes. The old version
% is still available for reference as oldplainhscode.

\newenvironment{plainhscode}%
  {\hsnewpar\abovedisplayskip
   \advance\leftskip\mathindent
   \hscodestyle
   \let\hspre\(\let\hspost\)%
   \pboxed}%
  {\endpboxed%
   \hsnewpar\belowdisplayskip
   \ignorespacesafterend}

\newenvironment{oldplainhscode}%
  {\hsnewpar\abovedisplayskip
   \advance\leftskip\mathindent
   \hscodestyle
   \let\\=\@normalcr
   \(\pboxed}%
  {\endpboxed\)%
   \hsnewpar\belowdisplayskip
   \ignorespacesafterend}

% Here, we make plainhscode the default environment.

\newcommand{\plainhs}{\sethscode{plainhscode}}
\newcommand{\oldplainhs}{\sethscode{oldplainhscode}}
\plainhs

% The arrayhscode is like plain, but makes use of polytable's
% parray environment which disallows page breaks in code blocks.

\newenvironment{arrayhscode}%
  {\hsnewpar\abovedisplayskip
   \advance\leftskip\mathindent
   \hscodestyle
   \let\\=\@normalcr
   \(\parray}%
  {\endparray\)%
   \hsnewpar\belowdisplayskip
   \ignorespacesafterend}

\newcommand{\arrayhs}{\sethscode{arrayhscode}}

% The mathhscode environment also makes use of polytable's parray 
% environment. It is supposed to be used only inside math mode 
% (I used it to typeset the type rules in my thesis).

\newenvironment{mathhscode}%
  {\parray}{\endparray}

\newcommand{\mathhs}{\sethscode{mathhscode}}

% texths is similar to mathhs, but works in text mode.

\newenvironment{texthscode}%
  {\(\parray}{\endparray\)}

\newcommand{\texths}{\sethscode{texthscode}}

% The framed environment places code in a framed box.

\def\codeframewidth{\arrayrulewidth}
\RequirePackage{calc}

\newenvironment{framedhscode}%
  {\parskip=\abovedisplayskip\par\noindent
   \hscodestyle
   \arrayrulewidth=\codeframewidth
   \tabular{@{}|p{\linewidth-2\arraycolsep-2\arrayrulewidth-2pt}|@{}}%
   \hline\framedhslinecorrect\\{-1.5ex}%
   \let\endoflinesave=\\
   \let\\=\@normalcr
   \(\pboxed}%
  {\endpboxed\)%
   \framedhslinecorrect\endoflinesave{.5ex}\hline
   \endtabular
   \parskip=\belowdisplayskip\par\noindent
   \ignorespacesafterend}

\newcommand{\framedhslinecorrect}[2]%
  {#1[#2]}

\newcommand{\framedhs}{\sethscode{framedhscode}}

% The inlinehscode environment is an experimental environment
% that can be used to typeset displayed code inline.

\newenvironment{inlinehscode}%
  {\(\def\column##1##2{}%
   \let\>\undefined\let\<\undefined\let\\\undefined
   \newcommand\>[1][]{}\newcommand\<[1][]{}\newcommand\\[1][]{}%
   \def\fromto##1##2##3{##3}%
   \def\nextline{}}{\) }%

\newcommand{\inlinehs}{\sethscode{inlinehscode}}

% The joincode environment is a separate environment that
% can be used to surround and thereby connect multiple code
% blocks.

\newenvironment{joincode}%
  {\let\orighscode=\hscode
   \let\origendhscode=\endhscode
   \def\endhscode{\def\hscode{\endgroup\def\@currenvir{hscode}\\}\begingroup}
   %\let\SaveRestoreHook=\empty
   %\let\ColumnHook=\empty
   %\let\resethooks=\empty
   \orighscode\def\hscode{\endgroup\def\@currenvir{hscode}}}%
  {\origendhscode
   \global\let\hscode=\orighscode
   \global\let\endhscode=\origendhscode}%

\makeatother
\EndFmtInput
%
%
%
\ReadOnlyOnce{colorcode.fmt}%

\RequirePackage{colortbl}
\RequirePackage{calc}

\makeatletter
\newenvironment{colorhscode}%
  {\hsnewpar\abovedisplayskip
   \hscodestyle
   \tabular{@{}>{\columncolor{codecolor}}p{\linewidth}@{}}%
   \let\\=\@normalcr
   \(\pboxed}%
  {\endpboxed\)%
   \endtabular
   \hsnewpar\belowdisplayskip
   \ignorespacesafterend}

\newenvironment{tightcolorhscode}%
  {\hsnewpar\abovedisplayskip
   \hscodestyle
   \tabular{@{}>{\columncolor{codecolor}\(}l<{\)}@{}}%
   \pmboxed}%
  {\endpmboxed%
   \endtabular
   \hsnewpar\belowdisplayskip
   \ignorespacesafterend}

\newenvironment{barhscode}%
  {\hsnewpar\abovedisplayskip
   \hscodestyle
   \arrayrulecolor{codecolor}%
   \arrayrulewidth=\coderulewidth
   \tabular{|p{\linewidth-\arrayrulewidth-\tabcolsep}@{}}%
   \let\\=\@normalcr
   \(\pboxed}%
  {\endpboxed\)%
   \endtabular
   \hsnewpar\belowdisplayskip
   \ignorespacesafterend}
\makeatother

\def\colorcode{\columncolor{codecolor}}
\definecolor{codecolor}{rgb}{1,1,.667}
\newlength{\coderulewidth}
\setlength{\coderulewidth}{3pt}

\newcommand{\colorhs}{\sethscode{colorhscode}}
\newcommand{\tightcolorhs}{\sethscode{tightcolorhscode}}
\newcommand{\barhs}{\sethscode{barhscode}}

\EndFmtInput

%%%%%%%%%%%%%%%%%%%%%%%%%%%%%%
%% 
%% Haskell Styling
%%
%% TODO: Figure out spacing!

%% Colors (from duo-tone light syntax)
\definecolor{hsblack}{RGB}{45,32,3}
\definecolor{hsgold1}{RGB}{179,169,149}
\definecolor{hsgold2}{RGB}{177,149,90}
\definecolor{hsgold3}{RGB}{190,106,13}%{192,96,4}%{132,97,19}
\definecolor{hsblue1}{RGB}{173,176,182}
\definecolor{hsblue2}{RGB}{113,142,205}
\definecolor{hsblue3}{RGB}{0,33,132}
\definecolor{hsblue4}{RGB}{97,108,132}
\definecolor{hsblue5}{RGB}{34,50,68}
\definecolor{hsred2}{RGB}{191,121,103}
\definecolor{hsred3}{RGB}{171,72,46}

%% LaTeX Kerning nastiness. By using curly braces to delimit color group,
%% it breaks spacing. The following seems to work:
%%
%% https://tex.stackexchange.com/questions/85033/colored-symbols/85035#85035
%%
\newcommand*{\mathcolor}{}
\def\mathcolor#1#{\mathcoloraux{#1}}
\newcommand*{\mathcoloraux}[3]{%
  \protect\leavevmode
  \begingroup
    \color#1{#2}#3%
  \endgroup
}
\newcommand{\HSKeyword}[1]{\mathcolor{hsgold3}{\textbf{#1}}}
\newcommand{\HSNumeral}[1]{\mathcolor{hsred3}{#1}}
\newcommand{\HSChar}[1]{\mathcolor{hsred2}{#1}}
\newcommand{\HSString}[1]{\mathcolor{hsred2}{#1}}
\newcommand{\HSSpecial}[1]{\mathcolor{hsblue4}{#1}}
\newcommand{\HSSym}[1]{\mathcolor{hsblue4}{#1}}
\newcommand{\HSCon}[1]{\mathcolor{hsblue3}{\mathit{#1}}}
\newcommand{\HSVar}[1]{\mathcolor{hsblue5}{\mathit{#1}}}
\newcommand{\HSComment}[1]{\mathcolor{hsgold2}{\textit{#1}}}


%%% lhs2TeX parser does not recognize '*' 
%%% in kind annotations, it thinks it is a multiplication.



\usepackage{ucs}
\usepackage[utf8x]{inputenc}
\usepackage{autofe}
\usepackage{textcomp}

%% Haskell snippet 
\newenvironment{myhaskell}
{
  \vspace{-0.35cm}
  \begin{center}
}
{
  \end{center}
  \vspace{-0.35cm}
}

%% Haskell snippet 
\newenvironment{myhaskellnv}
{
  \begin{center}
}
{
  \end{center}
}


\title{Thesis title}

\author{C.R. van der Rest}

\supervisor{Dr. W.S. Swierstra \\ Dr. M.M.T. Chakravarty \\ Dr. A. Serrano Mena }

\begin{document}
\maketitle

%% Set up the front matter of our book
\frontmatter
\tableofcontents

\chapter{Declaration}
Thanks to family, supervisor, friends and hops!
\\ \\
I declare that this thesis has been composed solely by myself and that it has not been
submitted, in whole or in part, in any previous application for a degree. Except where
stated otherwise by reference or acknowledgment, the work presented is entirely my
own.

\chapter{Abstract}
Abstract

%% Starts the mainmatter
\mainmatter

\chapter{Introduction}
This thesis concerns itself with the generation of complex test data in the context of property based testing specifically, and generic programming for indexed datatypes in general. 

\section{Motivation}

\section{Research Question and Goals}

  This thesis aims to work towards an answer to the following question: 

  \begin{center} \emph{
    How can we obtain constrained test data by generically deriving enumeration and/or sampling mechanisms for indexed datatypes?
  }\end{center}

  In addition to a theoretical exploration of this question, we intend to supply the following deliverables:

  \begin{itemize}
    \item A formalization in Agda of the various type universe explored in this thesis.
    
    \item A haskell library that demonstrates how the ideas developed in this thesis can be applied in a more practical setting. 
  \end{itemize}

\section{Contributions}

\section{Thesis Structure}

  This thesis is structured as follows: in chapter 2 we discuss some relevant 
  theoretical background, and in chapter 3 we describe some of the work related to 
  this thesis. Chapter 4 sketches the design of our generator combinator library. 
  Chapter 5 through 7 describe various type universes, and show how we may derive 
  generators for any type in those universes. Additionally, we sketch how we may prove 
  that the associated enumerations are complete. Chapter 8 and 9 are concerned with 
  how we can implement these ideas in Haskell, and provide a comprehensive framework 
  for the generation of well formed programs. Finally, chapter 10 provides a 
  conclusion and lists some of the possible future work.

\section{Methodology}

\subsection{Agda Model}

\subsection{Haskell Library}

\subsection{Notational Conventions}

\chapter{Background}
In this section, we will briefly discuss some of the relevant theoretical background 
for this thesis. We assume the reader to be familiar with the general concepts of both 
Haskell and Agda, as well as functional programming in general. We shortly touch upon 
the following subjects:

\begin{itemize}
  \item
  \emph{Type theory} and its relationship with \emph{classical logic} through the 
  \emph{Curry-Howard correspondence}

  \item 
  Some of the more advanced features of the programming language \emph{Agda}, which we 
  use for the formalization of our ideas: \emph{Codata}, \emph{Sized Types} and \emph
  {Universe Polymorphism}. 

  \item 
  \emph{Datatype generic programming} using \emph{type universes} and the design 
  patterns associated with datatype generic programming.  
\end{itemize}

  We present this section as a courtesy to those readers who might not be familiar 
  with these topics; anyone experienced in these areas should feel free to skip ahead. 

\section{Type Theory}

  \emph{Type theory} is the mathematical foundation that underlies the \emph{type 
  systems} of many modern programming languages. In type theory, we reason about \emph\
  {terms} and their \emph{types}. We briefly introduce some basic concepts, and show 
  how they relate to our proofs in Agda. 

  \subsection{Intuitionistic Type Theory}

  In Intuitionistic type theory consists of terms, types and judgements $a : A$ 
  stating that terms have a certain type. Generally we have the following two finite 
  constructions: $\mathbb{0}$ or the \emph{empty type}, containing no terms, and 
  $\mathbb{1}$ or the \emph{unit type} which contains exactly $1$ term. Additionally,
  the \emph{equality type}, $=$, captures the notion of equality for both terms and 
  types. The equalit type is constructed from \emph{reflexivity}, i.e. it is 
  inhabited by one term $refl$ of the type $a = a$. 

  Types may be combined using three constructions. The \emph{function type}, $a 
  \rightarrow b$ is inhabited by functions that take an element of type $a$ as input 
  and produce something of type $b$. The \emph{sum type}, $a + b$ creates a type that 
  is inhabited by \emph{either} a value of type $a$ \emph{or} a 
  value of type $b$. The \emph{product type}, $a * b$, is inhabited by a pair of 
  values, one of type $a$ and one of type $b$. In terms of set theory, these 
  operations correspond respectively to functions, \emph{cartesian product} and \emph
  {tagged union}. 

  \subsection{The Curry-Howard Equivalence}

  The \emph{Curry-Howard equivalence} establishes an isomorphism between \emph
  {propositions and types} and \emph{proofs and terms} \cite{wadler2015propositions}. 
  This means that for any type there is a corresponding proposition, and any term 
  inhabiting this type corresponds to a proof of the associated proposition. Types and 
  propositions are generally connected using the mapping shown in \cref{tbl:chiso}.

\begin{table}[h]\label{tbl:chiso}
\begin{center}\begin{framed}
\begin{tabular}{ll}
\multicolumn{1}{c}{\textbf{Classical Logic}} & \textbf{Type Theory} \\ \hline \hline
False                                        & $\bot$               \\
True                                         & $\top$               \\
$P \vee Q$                                   & $P + Q$              \\
$P \wedge Q$                                 & $P * Q$              \\
$p \Rightarrow Q$                            & $P \rightarrow Q$                       
\end{tabular}
\caption{Correspondence between classical logic and type theory}
\end{framed}\end{center}
\end{table}

  \begin{example}

    We can model the proposition $P \wedge (Q \vee R) \Rightarrow (P \wedge Q) \vee (P 
    \wedge R)$ as a function with the following type: 

\includeagdanv{2}{tautologytype}

    We can then prove that this implication holds on any proposition by supplying a 
    definition that inhabits the above type: 

\includeagda{2}{tautologydef}

  \end{example}

  In general, we may prove any proposition that captured as a type by writing a 
  programin that inhabits that type. Allmost all constructs are readily translatable 
  from proposition logic, except boolean negation, for which there is no corresponding 
  construction in type theory. Instead, we model negation using functions to the empty 
  type $\bot$. That is, we can prove a property $P$ to be false by writing a function \
  $P \rightarrow \bot$. This essentially says that $P$ is true, we can derive a \
  contradiction, hence it must be false. Alowing us to prove many properties including negation. 
  
  \begin{example}

    For example, we might prove that a property 
    cannot be both true and false, i.e. $\forall\ P\ .\ \neg(P \wedge \neg P)$: 

\includeagdanv{2}{notpandnotp}

  \end{example}

  However, there are properties of classical logic which do not carry over well 
  through the Curry-Howard isomorphism. A good example of this is the \emph{law of 
  excluded middle}, which cannot be proven in type theory: 

\includeagda{2}{excludedmiddle}

  This implies that type theory is incomplete as a proof system, in the sense that 
  there exist properties wich we cannot prove, nor disprove. 

\subsection{Dependent Types}

  Dependent type theory allows the definition of types that depend on values. In 
  addition to the constructs introduced above, one can use so-called $\Pi$-types and 
  $\Sigma$-types. 
  $\Pi$-types capture the idea of \emph{dependent function types}, that is, functions 
  whose output type may depend on the values of its input. Given some type $A$ and a 
  family $P$ of types indexed by values of type $A$ (i.e. $P$ has type $A \rightarrow 
  Type$), $\Pi$-types have the following form: 

\begin{equation*}
\Pi_{(x : A)} P(x) \equiv (x : A) \rightarrow P(x) 
\end{equation*}

  In a similar spirit, $\Sigma$-types are ordered \textit{pairs} of which the type
  of the second value may depend on te first value of the pair:

\begin{equation*}
\Sigma_{(x : A)} P(x) \equiv (x : A) \times P(x) 
\end{equation*}

  The Curry-Howard equivalence extends to $\Pi$- and $\Sigma$-types as well: they 
  can be used to model universal and existential quantification \cite
  {wadler2015propositions} (\cref{chisodependent}).

\begin{table}[h]\label{tbl:chisodependent}
\begin{center}\begin{framed}
\begin{tabular}{ll}
\multicolumn{1}{c}{\textbf{Classical Logic}} & \textbf{Type Theory}    \\ \hline \hline
$\exists\ x\ .\ P\ x$                        & $\Sigma_{(x : A)} P(x)$ \\
$\forall\ x\ .\ P\ x$                        & $\Pi_{(x : A)} P(x)$                    
\end{tabular}
\caption{Correspondence between quantifiers in classical logic and type theory}
\end{framed}\end{center}
\end{table}

  \begin{example} 
  
    we might capture the relation between universal and negated existential 
    quantification ($\forall\ x\ .\ \neg P\ x \Rightarrow \neg \exists\ x\ .\ P\ x$) 
    as follows: 

\includeagdanv{2}{forallnottonotexists} 

  \end{example}

  The correspondence between dependent pairs and existential quantification quite \
  beautifullly illustrates the constructive nature of proofs in type theory; we prove 
  any existential property by presenting a term together with a proof that the 
  required property holds for that term. 

\section{Agda}

  Agda is a programming language based on Intuitionistic type theory\cite
  {norell2008dependently}. Its syntax is broadly similar to Haskell's, though Agda's 
  type system is arguably more expressive, since types may depend on term level 
  values. 

  Due to the aforementioned correspondence between types and propositions, any Agda 
  program we write is simultaneously a proof of the proposition associated with its 
  type. Through this mechanism, Agda serves a dual purpose as a proof assistent. 

\subsection{Codata and Sized Types}\label{codata}

  All definitions in Agda are required to be \textit{total}, meaning that they must be 
  defined on all possible inputs, produce a result in finite time. To enforce this 
  requirement, Agda needs to check whether the definitions we provide are terminating. 
  As stated by the \emph{Halting Problem}, it is not possible to define a general 
  procedure to perform this check. Instead, Agda uses a \emph{sound approximation}, in 
  which it requires at least one argument of any recursive call to be \emph
  {syntactically smaller} than its corresponding function argument. A consequence of 
  this approach is that there are Agda programs that terminate, but are rejected by 
  the termination checker. This means that we cannot work with infinite data in the 
  same way as in the same way as in Haskell, which does not care about termination. 
  
  \begin{example}

    The following definition is perfectly fine in Haskell: 

\begin{myhaskellnv}
\begin{hscode}\SaveRestoreHook
\column{B}{@{}>{\hspre}l<{\hspost}@{}}%
\column{E}{@{}>{\hspre}l<{\hspost}@{}}%
\>[B]{}\HSVar{nats}\HSSym{::}\HSSpecial{\HSSym{[\mskip1.5mu} }\HSCon{Int}\HSSpecial{\HSSym{\mskip1.5mu]}}{}\<[E]%
\\
\>[B]{}\HSVar{nats}\HSSym{\mathrel{=}}\HSNumeral{0}\HSCon{\mathbin{:}}\HSVar{map}\;\HSSpecial{(}\HSSym{+}\HSNumeral{1}\HSSpecial{)}\;\HSVar{nats}{}\<[E]%
\ColumnHook
\end{hscode}\resethooks
\end{myhaskellnv}

    Meanwhile, an equivalent definition in Agda gets rejected by the Termination 
    checker. The recursive call to \ensuremath{\HSVar{nats}} has no arguments, so none are structurally 
    smaller, thus the termination checker flags this call.  

\includeagda{2}{natsnonterminating}

  \end{example}

  However, as long as we use \ensuremath{\HSVar{nats}} sensibly, there does not need to be a problem. 
  Nonterminating programs only arise with improper use of such a definition, for 
  example by calculating the length of \ensuremath{\HSVar{nats}}. We can prevent the termination 
  checker from flagging these kind of operations by making the lazy semantics 
  explicit, using \textit{codata} and {sized types}. Codata is a general term for 
  possible inifinite data, often described by a co-recursive definition. Sized types 
  extend the space of function definitions that are recognized by the termination 
  checker as terminating by tracking information about the size of values in types 
  \cite{abel2010miniagda}. In the case of lists, this means that we explicitly 
  specify that the recursive argument to the \ensuremath{\HSSym{\anonymous} \HSSym{∷\char95 }} constructor is a \textit{Thunk}, 
  which should only be evaluated when needed: 

\includeagda{2}{colist}

  We can now define \ensuremath{\HSVar{nats}} in Agda by wrapping the recursive call in a thunk, 
  explicitly marking that it is not to be evaluated until needed.  

\includeagda{2}{natsterminating}

  Since colists are possible infinite structures, there are some functions we can 
  define on lists, but not on colists. 
    
  \begin{example} Consider a function that attempts to calculate the length of a \ensuremath{\HSCon{Colist}}: 

\includeagdanv{2}{lengthdef}

    In this case \ensuremath{\HSVar{length}} is not accepted by the termination checker because the input 
    colist is indexed with size \ensuremath{\HSSym{∞}}, meaning that there is no finite upper bound on 
    its size. Hence, there is no guarantee that our function terminates when 
    inductively defined on the input colist. 

  \end{example}
  
  There are some cases in which we can convince the termination checker that our definition is terminating by using sized types. Consider the folowing function that increments every element in a list of naturals with its position: 

\includeagda{2}{incposdef}

  The recursive call to \ensuremath{\HSVar{incpos}} gets flagged by the termination checker; we know 
  that \ensuremath{\HSVar{map}} does not alter the length of a list, but the termination checker cannot 
  see this. For all it knows \ensuremath{\HSVar{map}} equals \ensuremath{\HSVar{const}\;\HSSpecial{\HSSym{[\mskip1.5mu} }\HSNumeral{1}\HSSpecial{\HSSym{\mskip1.5mu]}}}, which would make \ensuremath{\HSVar{incpos}} 
  non-terminating. The size-preserving property of \ensuremath{\HSVar{map}} is not reflected in its 
  type. To mitigate this issue, we can define an alternative version of the \ensuremath{\HSCon{List}} 
  datatype indexed with \ensuremath{\HSCon{Size}}, which tracks the depth of a value in its type. 

\includeagda{2}{sizedlistdef}

  Here \ensuremath{\HSSym{↑}\HSVar{i}} means that the depth of a value constructed using the $::$ constructor 
  is one deeper than its recursive argument. Incidently, the recursive depth of a 
  list is equal to its size (or length), but this is not necessarily the case. By 
  indexing values of \ensuremath{\HSCon{List}} with their size, we can define a version of \ensuremath{\HSVar{map}} which 
  reflects in its type that the size of the input argument is preserved: 

\includeagda{2}{sizedmapdef}

  Using this definition of \ensuremath{\HSVar{map}}, the definition of \ensuremath{\HSVar{incpos}} is no longer rejected 
  by the termination checker. 

\subsection{Universe Polymorphism}

  Contrary to Haskell, Agda does not have separate notions for \emph{types}, 
  \emph{kinds} and \emph{sorts}. Instead it provides an infinite hierarchy of 
  type universes, where level is a member of the next, i.e. \ensuremath{\HSCon{Set}\;\HSVar{n}\HSCon{\mathbin{:}}\HSCon{Set}\;\HSSpecial{(}\HSVar{n}\HSSym{+}\HSNumeral{1}\HSSpecial{)}}. 
  Agda uses this construction in favor of simply declaring \ensuremath{\HSCon{Set}\HSCon{\mathbin{:}}\HSCon{Set}} to avoid 
  the construction of contradictory statements through Russel's paradox. 

  This implies that every construction in Agda that ranges over some \ensuremath{\HSCon{Set}\;\HSVar{n}} can 
  only be used for values that are in \ensuremath{\HSCon{Set}\;\HSVar{n}}. It is not possible to define, for 
  example, a \ensuremath{\HSCon{List}} datatype that may contain both \emph{values} and \emph{types}
   for this reason. 

   We can work around this limitation by defining a \emph{universe polymorphic} 
   construction for lists: 

\includeagda{2}{upolylist}

  Allowing us to capture lists of types (such as \ensuremath{\HSCon{ℕ}\HSSym{∷}\HSCon{Bool}\HSSym{∷}\HSSpecial{\HSSym{[\mskip1.5mu} }\HSSpecial{\HSSym{\mskip1.5mu]}}}) and lists of 
  values (such as \ensuremath{\HSNumeral{1}\HSSym{∷}\HSNumeral{2}\HSSym{∷}\HSSpecial{\HSSym{[\mskip1.5mu} }\HSSpecial{\HSSym{\mskip1.5mu]}}}) using a single datatype. Agda allows for the 
  programmer to declare that \ensuremath{\HSCon{Set}\HSCon{\mathbin{:}}\HSCon{Set}} using the \ensuremath{\mbox{\enskip\{-\# OPTIONS --type-in-type  \#-\}\enskip}} 
  pragma. Throughout the development accompanying this thesis, we will refrain from 
  using this pragma wherever possible. The examples included in this thesis are often 
  not universe-polymorphic, since the universe level variables required often pollute 
  the code, and obfuscate the concept we are trying to convey. 

\section{Generic Programming and Type Universes}

  In \emph{Datatype generic programming}, we define functionality not for individual 
  types, but rather by induction on \emph{structure} of types. This means that generic 
  functions will not take values of a particular type as input, but a \emph{code} that 
  describes the structure of a type. Haskell's \ensuremath{\HSKeyword{deriving}} mechanism is a prime example 
  of this mechanism. Anytime we add \ensuremath{\HSKeyword{deriving}\;\HSCon{Eq}} to a datatype definition, GHC will, 
  in the background, convert our datatype to a structural representation, and use a 
  \emph{generic equality} to create 
  an instance of the \ensuremath{\HSCon{Eq}} typeclass for our type. 

\subsection{Design Pattern}\label{sec:tudesignpattern}

  Datatype generic programming often follows a common design pattern that is 
  independent of the structural representation of types involved. In general 
  we follow the following steps: 

  \begin{enumerate}
    \item
      First, we define a datatype \ensuremath{\HSCon{𝓤}} representing the structure of types, 
      often called a \emph{Universe}. 
    \item 
      Next, we define a semantics \ensuremath{\HSSym{⟦\char95 ⟧}\HSCon{\mathbin{:}}\HSCon{𝓤}\HSSym{→}\HSCon{K}} that associates codes in \ensuremath{\HSCon{𝓤}} 
      with an appropriate value of kind \ensuremath{\HSCon{K}}. In practice this is often a functorial 
      representation of kind \ensuremath{\HSCon{Set}\HSSym{→}\HSCon{Set}}.
    \item 
      Finally, we (often) define a fixed point combinator of type \ensuremath{\HSSpecial{(}\HSVar{u}\HSCon{\mathbin{:}}\HSCon{𝓤}\HSSpecial{)}\HSSym{→}\HSCon{Set}} 
      that calculates the fixpoint of \ensuremath{\HSSym{⟦}\HSVar{u}\HSSym{⟧}}. 
  \end{enumerate}

  This imposes the implicit requirement that if we want to represent some type 
  \ensuremath{\HSCon{T}} with a code \ensuremath{\HSVar{u}\HSCon{\mathbin{:}}\HSCon{𝓤}}, the fixpoint of \ensuremath{\HSVar{u}} should be isomorphic to \ensuremath{\HSCon{T}}. 

  Given these ingredients we have everything we need at hand to write generic 
  functions. Section $3$ of Ulf Norell's \emph{Dependently Typed Programming 
  in Agda} \cite{norell2008dependently} contains an in depth explanation of 
  how this can be done in Agda. We will only give a rough sketch of the most 
  common design pattern here. In general, a datatype generic function is supplied
  with a code \ensuremath{\HSVar{u}\HSCon{\mathbin{:}}\HSCon{𝓤}}, and returns a function whose type is dependent on the 
  code it was supplied with. 
  
  \begin{example}

    Suppose we are defining a generic procedure for decidable equality. We might use the following type signature for such a procedure:

\includeagdanv{2}{eqdef}

    If we now define \ensuremath{\HSSym{≟}} by induction over \ensuremath{\HSVar{u}}, we have a decision procedure 
    for decidable equality that may act on values on any type, provided their 
    structure can be described as a code in \ensuremath{\HSCon{𝓤}}. 

  \end{example}

\subsection{Example Universes}

  There exist many different type universes. We will give a short overview of 
  the universes used in this thesis here; they will be explained in more detail 
  later on when we define generic generators for them. The literature review in 
  \cref{sec:lituniverses} contains a brief discussion of type universes beyond 
  those used we used for generic enumeration. 

  \paragraph{Regular Types} 
    Although the universe of regular types is arguably 
    one of the simplest type universes, it can describe a wide variaty of 
    recursive algebraic datatypes \texttt{[citation]}, roughly corresponding to 
    the algebraic types in Haskell98. Examples of regular types are 
    \emph{natural numbers}, \emph{lists} and \emph{binary trees}. Regular types are 
    insufficient once we want to have a generic representation of mutually recursive 
    or indexed datatypes. 

  \paragraph{Indexed Containers}
    The universe of \emph{Indexed Containers} \cite{altenkirch2015indexed} 
    provides a generic representation of large class indexed datatypes by 
    induction on the index type. Datatypes we can describe using this universe 
    include \ensuremath{\HSCon{Fin}} (\cref{lst:deffin}), \ensuremath{\HSCon{Vec}} (\cref{lst:defvec}) and closed 
    lambda terms (\cref{lst:defwellscoped}).

  \paragraph{Indexed Descriptions}
    Using the universe of \emph{Indexed Descriptions} \cite{dagand2013cosmology}
    we can represent arbitrary indexed datatypes. This allows us to describe 
    datatypes that are beyond what can be described using indexed containers, 
    that is, datatypes with recursive subtrees that are interdependent or whose 
    recursive subtrees have indices that cannot be uniquely determined from the 
    index of a value. 

\chapter{Literature Review}
In this section, we discuss some of the existing literature that is relevant in the 
domain of generating test data for property based testing. We take a look at some 
existing testing libraries, techniques for generation of constrained test data, and a 
few type universes beyond those we used that aim to describe (at least a subset of) 
indexed datatypes. 

\section{Libraries for Property Based Testing}

  \textit{Property Based Testing} aims to assert properties that universally hold for 
  our programs by parameterizing tests over values and checking them against a 
  collection of test values. Libraries for property based testing often include some 
  kind of mechanism to automatically generate collections of test values. Existing 
  tools take different approaches towards generation of test data: \textit{QuickCheck} 
  \cite{claessen2011quickcheck} randomly generates values within the test domain, 
  while \textit{SmallCheck} \cite{runciman2008smallcheck} and \textit{LeanCheck} \cite
  {matela2017tools} exhaustively enumerate all values in the test domain up to a 
  certain point. There exist many libraries for property based testing. For brevity we 
  constrain ourselves here to those that are relevant in the domain of functional 
  programming and/or haskell. 

\subsection{QuickCheck} 

  Published in 2000 by Claessen \& Hughes \cite{claessen2011quickcheck}, QuickCheck 
  implements property based testing for Haskell. Test values are generated by sampling randomly from the domain of test values. QuickCheck supplies 
  the typeclass \texttt{Arbitrary}, whose instances are those types for which random 
  values can be generated. A property of type \ensuremath{\HSVar{a}\HSSym{\to} \HSCon{Bool}} can be tested if \ensuremath{\HSVar{a}} is an 
  instance of \texttt{Arbitrary}. Instances for most common Haskell types are supplied 
  by the library. If a property fails on a testcase, QuickCheck supplies a 
  counterexample. Consider the following faulty definition of \ensuremath{\HSVar{reverse}}: 

\begin{myhaskell}
\begin{hscode}\SaveRestoreHook
\column{B}{@{}>{\hspre}l<{\hspost}@{}}%
\column{17}{@{}>{\hspre}c<{\hspost}@{}}%
\column{17E}{@{}l@{}}%
\column{20}{@{}>{\hspre}l<{\hspost}@{}}%
\column{E}{@{}>{\hspre}l<{\hspost}@{}}%
\>[B]{}\HSVar{reverse}\HSSym{::}\HSCon{Eq}\;\HSVar{a}\HSSym{\Rightarrow} \HSSpecial{\HSSym{[\mskip1.5mu} }\HSVar{a}\HSSpecial{\HSSym{\mskip1.5mu]}}\HSSym{\to} \HSSpecial{\HSSym{[\mskip1.5mu} }\HSVar{a}\HSSpecial{\HSSym{\mskip1.5mu]}}{}\<[E]%
\\
\>[B]{}\HSVar{reverse}\;\HSSpecial{\HSSym{[\mskip1.5mu} }\HSSpecial{\HSSym{\mskip1.5mu]}}{}\<[17]%
\>[17]{}\HSSym{\mathrel{=}}{}\<[17E]%
\>[20]{}\HSSpecial{\HSSym{[\mskip1.5mu} }\HSSpecial{\HSSym{\mskip1.5mu]}}{}\<[E]%
\\
\>[B]{}\HSVar{reverse}\;\HSSpecial{(}\HSVar{x}\HSCon{\mathbin{:}}\HSVar{xs}\HSSpecial{)}{}\<[17]%
\>[17]{}\HSSym{\mathrel{=}}{}\<[17E]%
\>[20]{}\HSVar{nub}\;\HSSpecial{(}\HSSpecial{(}\HSVar{reverse}\;\HSVar{xs}\HSSpecial{)}\HSSym{\plus} \HSSpecial{\HSSym{[\mskip1.5mu} }\HSVar{x}\HSSpecial{,}\HSVar{x}\HSSpecial{\HSSym{\mskip1.5mu]}}\HSSpecial{)}{}\<[E]%
\ColumnHook
\end{hscode}\resethooks
\end{myhaskell}

  If we now test our function by calling \ensuremath{\HSVar{quickChec}\;\HSVar{reverse\char95 preserves\char95 length}}, we get the following output: 

\begin{tabbing}\ttfamily
~Test\char46{}QuickCheck\char62{}~quickCheck~reverse\char95{}preserves\char95{}length~\\
\ttfamily ~\char42{}\char42{}\char42{}~Failed\char33{}~Falsifiable~\char40{}after~8~tests~and~2~shrinks\char41{}\char58{}~~~~\\
\ttfamily ~\char91{}7\char44{}7\char93{}
\end{tabbing}

  We see that a counterexample was found after 8 tests \textit{and 2 shrinks}. Due to 
  the random nature of the tested values, the counterexamples that falsify a property 
  are almost never minimal counterexamples. QuickCheck takes a counterexample and 
  applies some function that produces a collection of values that are smaller than the 
  original counterexample, and attempts to falsify the property using one of the 
  smaller values. By repeatedly \textit{Shrinking} a counterexample, QuickCheck is 
  able to find much smaller counterexamples, which are in general of much more use to 
  the programmer. 

  Perhaps somewhat surprising is that QuickCheck is also able randomly generate values 
  for function types by modifying the seed of the random generator (which is used to 
  generate the function's output) based on it's input. 

\subsection{(Lazy) SmallCheck} 

  Contrary to QuickCheck, SmallCheck \cite{runciman2008smallcheck} takes an \textit
  {enumerative} approach to the generation of test data. While the approach to 
  formulation and testing of properties is largely similar to QuickCheck's, test 
  values are not generated at random, but rather exhaustively enumerated up to a 
  certain \textit{depth}. Zero-arity constructors have depth $0$, while the depth of 
  any positive arity constructor is one greather than the maximum depth of its 
  arguments. The motivation for this is the \textit{small scope hypothesis}: if a 
  program is incorrect, it will almost allways fail on some small input \cite
  {andoni2003evaluating}. 

  In addition to SmallCheck, there is also \textit{Lazy} SmallCheck. In many cases, 
  the value of a property is determined only by part of the input. Additionally, 
  Haskell's lazy semantics allow for functions to be defined on partial inputs. The 
  prime example of this is a property \texttt{sorted :: Ord a => [a] -> Bool} that 
  returns \texttt{false} when presented with \texttt{1:0:$\bot$}. It is not necessary 
  to evaluate $\bot$ to determine that the input list is not ordered. 

  Partial values represent an entire class of values. That is, \texttt{1:0:$\bot$} can 
  be viewed as a representation of the set of lists that have prefix \texttt{[1, 0]}. 
  By checking properties on partial values, it is possible to falsify a property for 
  an entire class of values in one go, in some cases greatly reducing the amount of 
  testcases needed. 

\subsection{LeanCheck} 

  Where SmallCheck uses a value's \textit{depth} to bound the number of test values, 
  LeanCheck uses a value's \textit{size} \cite{matela2017tools}, where size is defined 
  as the number of construction applications of positive arity. Both SmallCheck and 
  LeanCheck contain functionality to enumerate functions similar to QuickCheck's 
  \texttt{Coarbitrary}. 

\subsection{Hegdgehog} 
  
  Hedgehog \cite{hedgehog} is a framework similar to QuickCheck, that aims to be a 
  more modern alternative. It includes support for monadic effects in generators and 
  concurrent checking of properties. Additionally it supports automatic schrinking for many datatypes. Unlike QuickCheck and SmallCheck, HedgeHog does not support (partial) automatic derivation of generators, but rather chooses to supply a comprehensive set of combinators, which the user can then use to assemble generators.

\subsection{Feat} 
  
  A downside to both SmallCheck and LeanCheck is that they do not provide an efficient 
  way to generate or sample large test values. QuickCheck has no problem with either, 
  but QuickCheck generators are often more tedious to write compared to their 
  SmallCheck counterpart. Feat \cite{duregaard2013feat} aims to fill this gap by 
  providing a way to efficiently enumerate algebraic types, employing memoization 
  techniques to efficiently find the $n^{th}$ element of an enumeration. 

\subsection{QuickChick: QuickCheck for Coq} 
  
  QuickChick is a QuickCheck clone for the proof assistant Coq \cite
  {denes2014quickchick}. The fact that Coq is a proof assistant enables the user to 
  reason about the testing framework itself \cite{paraskevopoulou2015foundational}. 
  This allows one, for example, to prove that generators adhere to some distribution. 

\subsection{QuickSpec: Automatic Generation of Specifications}

  A surprising application of property based testing is the automatic generation of 
  program specifications, proposed by Claessen et al. \cite{claessen2010quickspec} 
  with the tool \textit{QuickSpec}. QuickSpec automatically generates a set of 
  candidate formal specifications given a list of pure functions, specifically in the 
  form of algebraic equations. Random property based testing is then used to falsify 
  specifications. In the end, the user is presented with a set of equations for which 
  no counterexample was found.  

\section{Generating Constrained Test Data}\label{genconstrainedtd}

  Defining a suitable generation of test data for property based testing potentially very challenging, independent of whether we choose to sample from 
  or enumerate the space of test values. Writing generators for mutually recursive 
  datatypes with a suitable distribution is especially challenging. 
    
  We run into prolems when we desire to generate test data for properties with a 
  precondition. If a property's precondition is satisfied by few input values, it 
  becomes unpractical to test such a property by simply generating random input data, 
  and using rejection sampling to filter out those values that satisfy the desired 
  precondition. We will often end up with very few testcases, and we will end up with 
  a skewed distribution favoring those test values that have the largest probability 
  to be picked at random (often these are the simplest values that satisfy the 
  precondition). 
  
  The usual solution to this problem is to define a custom test data generator that 
  only produces data that satisfies the precondition. There are cases in which this is 
  not too difficult, however once we require more complex test data, such as 
  well-formed programs, this is quite a challenging task. 

\subsection{Lambda Terms} 

  A problem often considered in literature is the generation of (well-typed) lambda 
  terms \cite{palka2011testing, grygiel2013counting, claessen2015generating}. Good 
  generation of arbitrary program terms is especially interesting in the context of 
  testing compiler infrastructure, and lambda terms provide a natural first step 
  towards that goal. 

  Claessen and Duregaard \cite{claessen2015generating} adapt the techniques described 
  by Duregaard \cite{duregaard2013feat} to allow efficient generation of constrained 
  data. They use a variation on rejection sampling, where the space of values is 
  gradually refined by rejecting classes of values through partial evaluation (similar 
  to SmallCheck \cite{runciman2008smallcheck}) until a value satisfying the imposed 
  constrained is found. 

  An alternative approach centered around the semantics of the simply typed lambda 
  calculus is described by Pa{\l}ka et al. \cite{palka2011testing}. Contrary to the 
  work done by Claessen and Duregaard \cite{claessen2015generating}, where 
  typechecking is viewed as a black box, they utilize definition of the typing rules 
  to devise an algorithm for generation of random lambda terms. The basic approach is 
  to take some input type, and randomly select an inference rule from the set of rules 
  that could have been applied to arrive at the goal type. Obviously, such a procedure 
  does not guarantee termination, as repeated application of the function application 
  rule will lead to an arbitrarily large goal type. As such, the algorithm requires a 
  maximum search depth and backtracking in order to guarantee that a suitable term 
  will eventually be generated, though it is not guaranteed that such a term exists if 
  a bound on term size is enforced \cite{moczurad2000statistical}. 

  Wang \cite{wang2005generating} considers the problem of generating closed untyped 
  lambda terms. 

\subsection{Inductive Relations in Coq}

  An approach to generation of constrained test data for Coq's QuickChick was proposed 
  by Lampropoulos et al. \cite{lampropoulos2017generating} in their 2017 paper \textit
  {Generating Good Generators for Inductive Relations}. They observe a common pattern 
  where the required test data is of a simple type, but constrained by some 
  precondition. The precondition is then given by some inductive dependent relation 
  indexed by said simple type. The \ensuremath{\HSCon{Sorted}} datatype shown in section \ref
  {introduction} is a good example of this

  They derive generators for such datatypes by abstracting over dependent inductive 
  relations indexed by simple types. For every constructor, the resulting type uses a 
  set of expressions as indices, that may depend on the constructor's arguments and 
  universally quantified variables. These expressions induce a set of unification 
  constraints that apply when using that particular constructor. These unification 
  constraints are then used when constructing generators to ensure that only values 
  for which the dependent inductive relation is inhabited are generated. 

\section{Generic Programming \& Type Universes}\label
{sec:lituniverses}

  Many type universes have been developed beyond those used in this thesis, some of 
  which are also designed to describe (a subset of) indexed datatypes. We describe a 
  few of them here, and briefly discuss how they relate to the universes we used. 

\subsection{SOP (Sum of Products)}\label{sop}

  On of the more simple representations is the so called \textit{Sum of Products} view 
  \cite{de2014true}, where datatypes are respresented as a choice between an arbitrary 
  amount of constructors, each of which can have any arity. This view corresponds to 
  how datatypes are defined in Haskell, and is closely related to the universe of 
  regular types. As we will see (for example in section \ref{patternfunctors}), other 
  universes too employ sum and product combinators to describe the structure of 
  datatypes, though they do not necessarily enforce the representation to be in 
  disjunctive normal form. Sum of Products, in its simplest form, cannot represent 
  mutually recursive families of datatypes. An extension that allows this has been 
  developed in \cite{miraldo2018sums}, and is available as a Haskell library through 
  \emph{Hackage}.  

\subsection{W-Types}\label{sec:wtypes}

  Introduced by Per Martin-Löf \cite{martin1984intuitionistic}, \emph{W-types} 
  abstract over tree-shaped data structures, such as natural numbers or binary trees. 
  W-types are defined by their \emph{shape} and \emph{position}, describing 
  respectively the set of constructors and the number of recursive positions. 

  Perhaps the best known definition of W-types is using an inductive datatype, with 
  one constructor taking a shape value, and a function from position to W-type: 

\includeagda{3}{winductive}

  However, we can use an alternate definition where we separate the universe into 
  codes, semantics and a fixpoint operation (listing \ref{lst:wtypes})

\includeagdalisting{3}{wtypes}{W-types defined with separate codes and semantics}
{lst:wtypes}

  We take this redundant step for two reasons: 

  \begin{enumerate}
    \item To unify the definition of W-types with the design pattern for type 
    universes we described in \cref{sec:tudesignpattern}. 

    \item To emphasize the the similarities between W-types, and the universe of 
    indexed containers, which will be further discussed in (TODO ref chapter 6)
  \end{enumerate}

  \begin{example} 

    Let us look at the natural numbers (listing \ref{lst:defnat}) as an example. We 
    can define the following W-type that is isomorphic to \ensuremath{\HSCon{ℕ}}:

\includeagdanv{3}{wnat}

    The \ensuremath{\HSCon{ℕ}} type has two constructors, hence our shape is a finite type with two 
    inhabitants (\ensuremath{\HSCon{Bool}} in this case). We then map \ensuremath{\HSVar{false}} to the empty type, 
    signifying that \ensuremath{\HSVar{zero}} has no recursive subtrees, and \ensuremath{\HSVar{true}} to the unit type, 
    denoting that \ensuremath{\HSVar{suc}} has one recursive subtree. The isomorphism between \ensuremath{\HSCon{ℕ}} and \ensuremath{\HSCon{Wℕ}}
     is established in listing \ref{wnatiso}. 

  \end{example}

\includeagdalisting{3}{wnatiso}{Isomorphism between \ensuremath{\HSCon{ℕ}} and \ensuremath{\HSCon{Wℕ}}}{lst:wnatiso}

\subsection{Indexed Functors}

  Löh and Magalhães propose in their paper \emph{Generic Programming with Indexed 
  Functors} \cite{loh2011generic} a type universe for generic programming in Agda, 
  that is able to handle a large class of indexed datatypes. Their universe takes the 
  universe of regular types as a basis. 
  
  The semantics of the universe, however, is not a functor \ensuremath{\HSCon{Set}\HSSym{→}\HSCon{Set}}, but rather an 
  \emph{indexed} functor \ensuremath{\HSSpecial{(}\HSCon{I}\HSSym{→}\HSCon{Set}\HSSpecial{)}\HSSym{→}\HSCon{O}\HSSym{→}\HSCon{Set}}. Additionally, they add some 
  combinators, such as first order constructors to encode isomorphisms and fixpoints 
  as part of their universe. 

\subsection{Combinatorial Species}

  Combinatorial Species. Combinatorial species \cite{yorgey2010species} were 
  originally developed as a mathematical framework, but can also be used as an 
  alternative way of looking at datatypes. A species can, in terms of functional 
  programming, be thought of as a type constructor with one polymorphic argument. 
  Haskell’s ADTs (or regular types in general) can be described by definining familiar 
  combinators for species, such as sum and product.

\chapter{A Combinator Library for Generators}
\section{The Type of Generators}

  We have not yet specified what it is exactly that we mean when we talk about \textit{generators}. In the context of property based testing, it makes sense to think of generators as entities that produce values of a certain type; the machinery that is responsible for supplying suitable test values. As we saw in section \cref{sec:literature}, this can mean different things depending on the library that you are using. \textit{SmallCheck} and \textit{LeanCheck} generators are functions that take a size parameter as input and produce an exhaustive list of all values that are smaller than the generator's input, while \textit{QuickCheck} generators randomly sample values of the desired type. Though various libraries use different terminology to refer to the mechanisms used to produce test values, we will use \textit{generator} as an umbrella term to refer to the test data producing parts of existing libraries. 

  \subsection{Examples in Existing Libraries}
  
   When comparing generator definitions across libraries, we see that their definition is often more determined by the structure of the datatype they ought to produce values of than the type of the generator itself. Let us consider the \ensuremath{\HSCon{Nat}} datatype (definition \ref{defnat}). In QuickCheck, we could define a generator for the \ensuremath{\HSCon{Nat}} datatype as follows: 

\begin{hscode}\SaveRestoreHook
\column{B}{@{}>{\hspre}l<{\hspost}@{}}%
\column{3}{@{}>{\hspre}l<{\hspost}@{}}%
\column{21}{@{}>{\hspre}l<{\hspost}@{}}%
\column{E}{@{}>{\hspre}l<{\hspost}@{}}%
\>[3]{}\HSVar{genNat}\HSSym{::}\HSCon{Gen}\;\HSCon{Nat}{}\<[E]%
\\
\>[3]{}\HSVar{genNat}\HSSym{\mathrel{=}}\HSVar{oneof}\;\HSSpecial{\HSSym{[\mskip1.5mu} }{}\<[21]%
\>[21]{}\HSVar{pure}\;\HSCon{Zero}\HSSpecial{,}\HSCon{Suc}\HSSym{<\$>}\HSVar{genNat}\HSSpecial{\HSSym{\mskip1.5mu]}}{}\<[E]%
\ColumnHook
\end{hscode}\resethooks

  QuickCheck includes many combinators to finetune the distribution of values of the generated type, which are omitted in this case since they do not structurally alter the generator. Compare the above generator to its SmallCheck equivalent: 

\begin{hscode}\SaveRestoreHook
\column{B}{@{}>{\hspre}l<{\hspost}@{}}%
\column{3}{@{}>{\hspre}l<{\hspost}@{}}%
\column{24}{@{}>{\hspre}c<{\hspost}@{}}%
\column{24E}{@{}l@{}}%
\column{28}{@{}>{\hspre}l<{\hspost}@{}}%
\column{E}{@{}>{\hspre}l<{\hspost}@{}}%
\>[B]{}\HSKeyword{instance}\;\HSCon{Serial}\;\HSVar{m}\;\HSCon{Nat}\;\HSKeyword{where}{}\<[E]%
\\
\>[B]{}\hsindent{3}{}\<[3]%
\>[3]{}\HSVar{series}\HSSym{\mathrel{=}}\HSVar{cons0}\;\HSCon{Zero}{}\<[24]%
\>[24]{}\HSSym{\char92 /}{}\<[24E]%
\>[28]{}\HSCon{Cons1}\;\HSCon{Suc}{}\<[E]%
\ColumnHook
\end{hscode}\resethooks

  Both generator definitions have a strikingly similar structure, marking a choice between the two available constructors (\ensuremath{\HSCon{Zero}} and \ensuremath{\HSCon{Suc}}) and employing a appropriate combinators to produce values for said constructors. Despite this structural similarity, the underlying types of the respective generators are wildly different, with \ensuremath{\HSVar{genNat}} being an \ensuremath{\HSCon{IO}} operation that samples random values and the \ensuremath{\HSCon{Serial}} instance being a function taking a depth and producing all values up to that depth. 

\subsection{Separating Structure and Interpretation}

  The previous example suggests that there is a case to be made for separating a generators structure from the format in which test values are presented. Additionally, by having a single datatype representing a generator's structure, we shift the burden of proving termination from a generator's definition to its interpretation, which in Agda is a considerable advantage. In practice this means that we define some datatype \ensuremath{\HSCon{Gen}\;\HSVar{a}} that marks the structure of a generator, and a function \ensuremath{\HSVar{interpret}\HSCon{\mathbin{:}}\HSCon{Gen}\;\HSVar{a}\HSSym{\to} \HSCon{T}\;\HSVar{a}} that maps an input structure to some \ensuremath{\HSCon{T}\;\HSVar{a}}, where \ensuremath{\HSCon{T}} which actually produces test values. In our case, we will almost exclusively consider an interpretation of generators to functions of type \ensuremath{\HSCon{ℕ}\HSSym{→}\HSCon{List}\;\HSVar{a}}, but we could have chosen \ensuremath{\HSCon{T}} to by any other type of collection of values of type \ensuremath{\HSVar{a}}. An implication of this separation is that, given suitable interpretation functions, a user only has to define a single generator in order to be able to employ different strategies for generating test values, potentially allowing for both random and enumerative testing to be combined into a single framework. 

  This approach means that generator combinators are not functions that operate on a a generator's result, such as merging two streams of values, but rather a constructor of some abstract generator type; \ensuremath{\HSCon{Gen}} in our case. This datatype represents generators in a tree-like structure, not unlike the more familiar abstract syntax trees used to represent parsed programs. 

\subsection{The \ensuremath{\HSCon{Gen}} Datatype}

  We define the datatype of generators, \ensuremath{\HSCon{Gen}\;\HSVar{a}\;\HSVar{t}}, to be a family of types indexed by two types \footnote{The listed definition will not be accepted by Agda due to inconsistencies in the universe levels. This is also the case for many code examples to come. To keep things readable, we will not concern ourselves with universe levels throughout this thesis.}. One signifying the type of values that are produced by the generator, and one specifying the type of values produced by recursive positions. 

\includeagdalisting{4}{gendef}{Definition of the \ensuremath{\HSCon{Gen}} datatype}{lst:gendef}

  \textit{Closed} generators are then generators produce that produce the values of the same type as their recursive positions: 

\includeagda{4}{gdef}

  The \ensuremath{\HSCon{Pure}} and \ensuremath{\HSCon{Ap}} constructors make \ensuremath{\HSCon{Gen}} an instance of \ensuremath{\HSCon{Applicative}}, meaning that we can (given a fancy operator for denoting choice) denote generators in way that is very similar to their definition: 

\includeagda{4}{gennat}

  This serves to emphasize that the structure of generators can, in the case of simpler datatypes, be mechanically derived from the structure of a datatype. We will see how this can be done in chapter \cref{chap:derivingregular}. 

  The question remains how to deal with constructors that refer to \textit{other} types. For example, consider the type of lists (definition \ref{deflist}). We can define an appropriate generator following the structure of the datatype definition: 

\includeagda{4}{listgenhole}

  It is however not immediately clear what value to supply to the remaining interaction point. If we inspect its goal type we see that we should supply a value of type \ensuremath{\HSCon{Gen}\;\HSVar{a}\;\HSSpecial{(}\HSCon{List}\;\HSVar{a}\HSSpecial{)}}: a generator producing values of type \ensuremath{\HSVar{a}}, with recursive positions producing values of type \ensuremath{\HSCon{List}\;\HSVar{a}}. This makes little sense, as we would rather be able to invoke other \textit{closed generators} from within a generator. To do so, we add another constructor to the \ensuremath{\HSCon{Gen}} datatype, that signifies the invokation of a closed generator for another datatype: 

\includeagda{4}{calldef}

  Using this definition of \ensuremath{\HSCon{Call}}, we can complete the previous definition for \ensuremath{\HSVar{list}}: 

\includeagda{4}{listgen}

\subsection{Generator Interpretations}

  We can view a generator's interpretation as any function mapping generators to some type, where the output type is parameterized by the type of values produced by a generator: 

\includeagda{4}{intdef}

  From this definition of \ensuremath{\HSCon{Interpretation}}, we can define concrete interpretations. For example, if we want to behave our generators similar to SmallCheck's \ensuremath{\HSCon{Series}}, we might define the following concrete instantiation of the \ensuremath{\HSCon{Interpretation}} type: 

\includeagda{4}{scdef}

  We can then define a generator's behiour by supplying a definition that inhabits the \ensuremath{\HSCon{GenAsList}} type: 

\includeagda{4}{scint}

  The goal type of the open interaction point is then $\mathbb{N}$\ensuremath{\HSSym{→}\HSCon{List}\;\HSVar{a}}. We will see in \cref{sec:enuminterpretation} how we can flesh out this particular interpretation. We could however have chosen any other result type, depending on what suits our particular needs. An alternative would be to interpret generators as a \ensuremath{\HSCon{Colist}}, omitting the depth bound altogether:

\includeagda{4}{intcolist}

\section{Generalization to Indexed Datatypes}

  A first approximation towards a generalization of the \ensuremath{\HSCon{Gen}} type to indexed types might be to simply lift the existing definition from \ensuremath{\HSCon{Set}} to \ensuremath{\HSCon{I}\HSSym{→}\HSCon{Set}}. 

\includeagda{4}{liftgen}

  However, by doing so we implicitly impose the constraint that the recursive positions of a value have the same index as the recursive positions within it. Consider, for example, the \ensuremath{\HSCon{Fin}} type (definition \ref{findef}). If we attempt to define a generator using the lifted type, we run into a problem. 

\includeagda{4}{finhole}

  Any attempt to fill the open interaction point with the \ensuremath{\HSVar{μ}} constructor fails, as it expects a value of \ensuremath{\HSCon{Gen}\;\HSSpecial{(}\HSCon{Fin}\;\HSVar{n}\HSSpecial{)}\;\HSSpecial{(}\HSCon{Fin}\;\HSVar{suc}\;\HSVar{n}\HSSpecial{)}}, but \ensuremath{\HSVar{μ}} requires both its type parameters to be equal. We can circumvent this issue by using direct recursion. 

\includeagda{4}{findirect}

  It is however clear that this approach becomes a problem once we attempt to define generators for datatypes with recursive positions which have indices that are not structurally smaller than the index they target. To overcome these limitations we resolve to a separate deep embedding of generators for indexed types. 

\includeagdalisting{4}{genidef}{Definitiong of the \ensuremath{\HSCon{Genᵢ}} datatype}{lst:genidef}

  And consequently the type of closed indexed generators. 

\includeagda{4}{gidef}

  Notice how the \ensuremath{\HSCon{Apᵢ}} constructor allows for its second argument to have a different index. The reason for this becomes clear when we 

  With the same combinators as used for the \ensuremath{\HSCon{Gen}} type, we can now define a generator for the \ensuremath{\HSCon{Fin}} type. 

\includeagda{4}{genfin}

  Now defining generators for datatypes with recursive positions whose indices are not structurally smaller than the index of the datatype itself can be done without complaints from the termination checker, such as well-scoped $\lambda$-terms (definition \ref{defwellscoped}). 

\includeagda{4}{wellscoped}

  It is important to note that it is not possible to call indexed generators from simple generators and vice versa with this setup. We can allow this by either parameterizing the \ensuremath{\HSCon{Call}} and \ensuremath{\HSVar{iCall}} constructors with the datatype they refer to, or by adding extra constructors to the \ensuremath{\HSCon{Gen}} and \ensuremath{\HSCon{Genᵢ}} datatypes, making them mutually recursive. 

\section{Interpreting Generators as Enumerations}\label{sec:enuminterpretation}

  We will now consider an example interpretation of generators where we map values of the \ensuremath{\HSCon{Gen}} or \ensuremath{\HSCon{Genᵢ}} datatypes to functions of type $\mathbb{N}$\ensuremath{\HSSym{→}\HSCon{List}\;\HSVar{a}}. The constructors of both datatypes mimic the combinators used Haskell's \ensuremath{\HSCon{Applicative}} and \ensuremath{\HSCon{Alternative}} typeclasses, so we can use the \ensuremath{\HSCon{List}} instances of these typeclasses for guidance when defining an enumerative interpretation.  

\includeagdalisting{4}{tolist}{Interpretation of the \ensuremath{\HSCon{Gen}} datatype as an enumeration}{lst:tolist}

  Similarly, we can define such an interpretation for the \ensuremath{\HSCon{Genᵢ}} datatype similar to listing \ref{lst:tolist} with the only difference being the appropriate indices getting passed to recursive calls. Notice how our generator's behaviour - most notably the intended semantics of the input depth bound - is entirely encoded within the definition of the interpretation. In this case by decrementing \ensuremath{\HSVar{n}} anytime a recursive position is encountered.  

\section{Properties for Enumerations}

\section{Generating Function Types}

\section{Monadic Generators}

  There are some cases in which the applicative combinators are not expressive enough to capture the desired generator. For example, if we were to define a construction for generation of $\Sigma$ types, we encounter some problems. 

\includeagda{4}{sigmagenhole}

  We can extend the \ensuremath{\HSCon{Gen}} datatype with a \ensuremath{\HSCon{Bind}} operation that mimics the monadic bind operator (\ensuremath{\HSSym{\bind} }) to allow for such dependencies to exist between generated values.

\includeagda{4}{sigmagen}

\chapter{Generic Generators for Regular types}

  A large class of recursive algebraic data types can be described with the universe 
  of \emph{regular types}. In this section we lay out this universe, together with its 
  semantics, and describe how we may define functions over regular types by induction 
  over their codes. We will then show how this allows us to derive from a code a 
  generic generator that produces all values of a regular type. We sketch how we can 
  prove that these generators are indeed complete. 

\section{The universe of regular types}

  Though the exact definition may vary across sources, the universe of regular types 
  is generally regarded to consist of the \emph{empty type} (or $\mathbb{0}$), the 
  unit type (or $\mathbb{1}$) and constants types. It is closed under both products 
  and coproducts \footnote{This roughly corresponds to datatypes in Haskell 98}. We 
  can define a datatype for this universe in Agda as shown in lising \ref{lst:regular}

\includeagdalisting{5}{regular}{The universe of regular types}{lst:regular}

  The semantics associated with the \ensuremath{\HSCon{Reg}} datatype, as shown in listing \ref
  {lst:regsem}, map a code to a functorial representation of a datatype, commonly 
  known as its \emph{pattern functor}. The datatype that is represented by a code is 
  isomorphic to the least fixpoint of its pattern functor. We fix pattern functors 
  using the following fixpoint combinator: 

\includeagda{5}{regularfix}

\includeagdalisting{5}{regularsemantics}{Semantics of the universe of regular types}
{lst:regsem}

  \begin{example}

    The type of natural numbers (see listing \ref{lst:defnat}) 
    exposes two constructors: the nullary constructor \ensuremath{\HSVar{zero}}, and the unary 
    constructor \ensuremath{\HSVar{suc}} that takes one recursive argument. We may thus view this type as 
    a coproduct (i.e. choice) of either a \emph{unit type} or a \emph{recursive 
    subtree}: 

\includeagdanv{5}{natregular}

    We convince ourselves that \ensuremath{\HSCon{ℕ'}} is indeed equivalent to \ensuremath{\HSCon{ℕ}} by defining conversion 
    functions, and showing their composition is extensionally equal to the identity 
    function, shown in listing \ref{lst:natiso}. 

  \end{example}

\includeagdalisting{5}{natiso}{Isomorphism between \ensuremath{\HSCon{ℕ}} and \ensuremath{\HSCon{ℕ'}}}{lst:natiso}

  We may then say that a type is regular if we can provide a proof that it is 
  isomorphic to the fixpoint of some \ensuremath{\HSVar{c}} of type \ensuremath{\HSCon{Reg}}. We use a record to capture 
  this notion, consisting of a code and an value that witnesses the isomorphism.

\includeagda{5}{regularrecord}

  By instantiating \ensuremath{\HSCon{Regular}} for a type, we may use any generic functionality that is defined over regular types. 

\subsection{Non-regular data types}

  Although there are many algebraic datatypes that can be described in the universe 
  of regular types, some cannot. Perhaps the most obvious limitation the is lack of 
  ability to caputure data families indexed with values. The regular univeres 
  imposes the implicit restriction that a datatype is uniform in the sens that all 
  recursive subtrees are of the same type. Indexed families, however, allow for 
  recursive subtrees to have a structure that is different from the structure of the 
  datatype they are a part of. 

  Furethermore, any family of mutually recursive datatypes cannot be described as a 
  regular type; again, this is a result of the restriction that recursive positions 
  allways refer to a datatype with the same structure. 

\section{Generic Generators for regular types}

  We can derive generators for all regular types by induction over their associated 
  codes. Furthermore, we will show in section \cref{regularproof} that, once 
  interpreted as enumerators, these generators are complete; i.e. any value will 
  eventually show up in the enumerator, provided we supply a sufficiently large size 
  parameter.  

\subsection{Defining functions over codes}

  If we apply the approach described in \cref{sec:tudesignpattern} without care, we 
  run into problems. Simply put, we cannot work with values of type \ensuremath{\HSCon{Fix}\;\HSVar{c}}, since 
  this implicitly imposes the restriction that any \ensuremath{\HSCon{I}} in \ensuremath{\HSVar{c}} refers to \ensuremath{\HSCon{Fix}\;\HSVar{c}}. 
  However, as we descent into recursive calls, the code we are working with changes, 
  and with it the type associated with recursive positions. For example: the \ensuremath{\HSCon{I}} in (\ensuremath{\HSCon{U}\HSSym{⊕}\HSCon{I}}) refers to values of type \ensuremath{\HSCon{Fix}\;\HSSpecial{(}\HSCon{U}\HSSym{⊕}\HSCon{I}\HSSpecial{)}}, not \ensuremath{\HSCon{Fix}\;\HSCon{I}}. We need to make a 
  distinction between the code we are currently working on, and the code that 
  recursive positions refer to. For this reason, we cannot define the generic 
  generator, \ensuremath{\HSVar{deriveGen}}, with the following type signature: 

\includeagda{5}{genericgen}

  If we observe that \ensuremath{\HSSym{⟦}\HSVar{c}\HSSym{⟧}\HSSpecial{(}\HSCon{Fix}\;\HSVar{c}\HSSpecial{)}\HSSym{≃}\HSCon{Fix}\;\HSVar{c}}, we may alter the type signature of \ensuremath{\HSVar{deriveGen}} slightly, such that it takes two input codes instead of one

\includeagda{5}{genericgen2}

  This allows us to induct over the first input code, while still being able to have 
  recursive positions reference the correct \emph{top-level code}. Notice that the 
  first and second type parameter of \ensuremath{\HSCon{Gen}} are different. This is intensional, as we 
  would otherwise not be able to use the $\mu$ constructor to mark recursive 
  positions.  

\subsection{Composing generic generators}

  Now that we have the correct type for \ensuremath{\HSVar{deriveGen}} in place, we can start defining 
  it. Starting with the cases for \ensuremath{\HSCon{Z}} and \ensuremath{\HSCon{U}}: 

\includeagda{5}{genericgenZU}

  Both cases are trivial. In case of the \ensuremath{\HSCon{Z}} combinator, we yield a generator that 
  produces no elements. As for the \ensuremath{\HSCon{U}} combinator, \ensuremath{\HSSym{⟦}\HSCon{U}\HSSym{⟧}\HSSpecial{(}\HSCon{Fix}\;\HSVar{c'}\HSSpecial{)}} equals \ensuremath{\HSSym{⊤}}, so we 
  need to return a generator that produces all inhabitants of \ensuremath{\HSSym{⊤}}. This is simply done 
  by lifting the single value \ensuremath{\HSVar{tt}} into the generator type. 

  In case of the \ensuremath{\HSCon{I}} combinator, we cannot simply use the $\mu$ constructor right 
  away. In this context, $\mu$ has the type \ensuremath{\HSCon{Gen}\;\HSSpecial{(}\HSSym{⟦}\HSVar{c'}\HSSym{⟧}\HSSpecial{(}\HSCon{Fix}\;\HSVar{c'}\HSSpecial{)}\HSSpecial{)}\;\HSSpecial{(}\HSSym{⟦}\HSVar{c'}\HSSym{⟧}\HSSpecial{(}\HSCon{Fix}\;\HSVar{c'}\HSSpecial{)}\HSSpecial{)}}
  . However, since \ensuremath{\HSSym{⟦}\HSCon{I}\HSSym{⟧}\HSSpecial{(}\HSCon{Fix}\;\HSVar{c}\HSSpecial{)}} equals \ensuremath{\HSCon{Fix}\;\HSVar{c}}, the types do not lign up. We need 
  to map the \ensuremath{\HSCon{In}} constructor over $\mu$ to fix this: 

\includeagda{5}{genericgenI}

  Moving on to products and coproducts: with the correct type for \ensuremath{\HSVar{deriveGen}} in place,
   we can define their generators quite easily by recursing on the left and right 
   subcodes, and combining their results using the appropriate generator combinators: 

\includeagda{5}{genericgenPCOP}

  Although defining \ensuremath{\HSVar{deriveGen}} constitutes most of the work, we are not quite there 
  yet. Since the the \ensuremath{\HSCon{Regular}} record expects an isomorphism with \ensuremath{\HSCon{Fix}\;\HSVar{c}}, we still 
  need to wrap the resulting generator in the \ensuremath{\HSCon{In}} constructor: 

\includeagda{5}{genericgenFinal}

  The elements produced by \ensuremath{\HSVar{genericGen}} can now readily be transformed into the 
  required datatype through an appropriate isomorphism. 

  \begin{example}

    We derive a generator for natural numbers by invoking \ensuremath{\HSVar{genericGen}} on the 
    appropriate code \ensuremath{\HSCon{U}\HSSym{⊕}\HSCon{I}}, and applying the isomorphism defined in listing \ref
    {natiso} to its results: 

\includeagdanv{5}{genericgenNat}

  \end{example}

  In general, we can derive a generator for any type \ensuremath{\HSCon{A}}, as long as there is an 
  instance argument of the type \ensuremath{\HSCon{Regular}\;\HSCon{A}} in scope: 

\includeagda{5}{isogen}

\section{Constant Types}

  In some cases, we describe datatypes as a compositions of other datatypes. An 
  example of this would be lists of numbers, \ensuremath{\HSCon{List}\;\HSCon{ℕ}}. Our current universe definition 
  is not expressive enough to do this. 
  
  \begin{example}

    Given the code representing natural numbers (\ensuremath{\HSCon{U}\HSSym{⊕}\HSCon{I}}) and lists (\ensuremath{\HSCon{U}\HSSym{⊕}\HSSpecial{(}\HSCon{C}\HSSym{⊗}\HSCon{I}\HSSpecial{)}}, 
    where \ensuremath{\HSCon{C}} is a code representing the type of elements in the list), we might be 
    tempted to try and replace \ensuremath{\HSCon{C}} with the code for natural numbers in the code for 
    lists: 

  \includeagdanv{5}{natlist}

    This code does not describe lists of natural numbers. The problem here is that the 
    two recursive positions refer to the \emph{same} code, which is incorrect. We need 
    the first \ensuremath{\HSCon{I}} to refer to the code of natural numbers, and the second \ensuremath{\HSCon{I}} to refer 
    to the entire code. 

  \end{example}

\subsection{Definition and Semantics}

  In order to be able to refer to other recursive datatypes, the universe of regular 
  types often includes a constructor marking \emph{constant types}: 

\includeagda{5}{constantdef}

  The \ensuremath{\HSCon{K}} constructor takes one parameter of type \ensuremath{\HSCon{Set}}, marking the type it 
  references. The semantics of \ensuremath{\HSCon{K}} is simply the type it carries: 

\includeagda{5}{constantsemantics}

  \begin{example}
    
    Given the addition of \ensuremath{\HSCon{K}}, we can now define a code that represents lists of 
    natural numbers: 

\includeagdanv{5}{natlist2}

    With the property that \ensuremath{\HSVar{listℕ}\HSSym{≃}\HSCon{List}\;\HSCon{ℕ}}. 

  \end{example}

\subsection{Generic Generators for Constant Typse}

  When attempting to define \ensuremath{\HSVar{deriveGen}} on \ensuremath{\HSCon{K}\;\HSVar{s}}, we run into a problem. We need to 
  return a generator that produces values of type \ensuremath{\HSVar{s}}, but we have no information 
  about \ensuremath{\HSVar{s}} whatsoever, apart from knowing that it lies in \ensuremath{\HSCon{Set}}. This is a problem, 
  since we cannot derive generators for arbitrary values in \ensuremath{\HSCon{Set}}. This leaves us with 
  two options: either we restrict the types that \ensuremath{\HSCon{K}} may carry to those types for 
  which we can generically derive a generator, or we require the programmer to supply 
  a generator for every constant type in a code. We choose the latter, since it has 
  the advantage that we can generate a larger set of types. 

  We have the programmer supply the necessary generators by defining a \emph{metadata} 
  structure, indexed by a code, that carries additional information for every \ensuremath{\HSCon{K}} 
  constructor used. We then parameterize \ensuremath{\HSVar{deriveGen}} with a metadata structure, 
  indexed by the code we are inducting over. The definition of the metadata structure 
  is shown in listing \ref{lst:mdstructure}. 

\includeagdalisting{5}{mdstructure}{Metadata structure carrying additional information 
for constant types}{lst:mdstructure}

  We then adapt the type of \ensuremath{\HSVar{deriveGen}} to accept a parameter containing the required 
  metadata structure: 

\includeagda{5}{derivegenKTy} 

  We then define \ensuremath{\HSVar{deriveGen}} as follows for constant types. All cases for existing 
  constructors remain the same. 

\includeagda{5}{derivegenKCase}

\section{Complete Enumerators For Regular Types}

  By applying the \ensuremath{\HSVar{toList}} interpretation shown in listing \ref{lst:tolist} to our 
  generic generator for regular types we obtain a complete enumeration for regular 
  types. Obviously, this relies on the programmer to supply complete generators for 
  all constant types referred to by a code. 

  We formulate the desired completeness property as follows: \textit{for every code c 
  and value x it holds that there is an n such that x occurs at depth n in the 
  enumeration derived from c}. In Agda, this amounts to proving the following 
  statement: 

\includeagda{5}{genericgencomplete}

  Just as was the case with deriving generators for codes, we need to take into the 
  account the difference between the code we are currently working with, and the top 
  level code. To this end, we alter the previous statement slightly. 

\includeagda{5}{derivegencomplete}

  If we invoke this lemma with two equal codes, we may leverage the fact that \ensuremath{\HSCon{In}} is 
  bijective to obtain a proof that \ensuremath{\HSVar{genericGen}} is complete too. The key observation 
  here is that mapping a bijective function over a complete generator results in 
  another complete generator. 

  The completeness proof roughly follows the following steps: 

  \begin{itemize}

    \item 
      First, we prove completeness for individual generator combinators 

    \item 
      Next, we assemble a suitable metadata structure to carry the required proofs 
      for constant types in the code. 

    \item 
      Finally, we assemble the individual components into a proof of the statement 
      above. 

  \end{itemize}

\subsection{Combinator Correctness}

  We start our proof by asserting that the used combinators are indeed complete. That 
  is, we show for every constructor of \ensuremath{\HSCon{Reg}} that the generator we return in \ensuremath{\HSVar{deriveGen}} produces all elements of the interpretation of that constructor. In the 
  case of \ensuremath{\HSCon{Z}} and \ensuremath{\HSCon{U}}, this is easy. 

\includeagda{5}{derivegencompleteZU}

  The semantics of \ensuremath{\HSCon{Z}} is the empty type, so any generator producing values of type \ensuremath{\HSSym{⊥}}
   is trivially complete. Similarly, in the case of \ensuremath{\HSCon{U}} we simply need to show that 
   interpreting \ensuremath{\HSVar{pure}\;\HSVar{tt}} returns a list containing \ensuremath{\HSVar{tt}}. 

  Things become a bit more interesting once we move to products and coproducts. In the 
  case of coproducts, we know the following equality to hold, by definition of both \ensuremath{\HSVar{toList}} and \ensuremath{\HSVar{deriveGen}}: 

\includeagda{5}{tolistcopeq}

  Basically, this equality unfolds the \ensuremath{\HSVar{toList}} function one step. Notice how the 
  generators on the left hand side of the equation are \emph{almost} the same as the 
  recursive calls we make. This means that we can prove completeness for coproducts by 
  proving the following lemmas, where we obtain the required completeness proofs by 
  recursing on the left and right subcodes of the coproduct. 

\includeagda{5}{mergecomplete}

  Similarly, by unfolding the toList function one step in the case of products, we get 
  the following equality:

\includeagda{5}{tolistpeq}

  We can prove the right hand side of this equality by proving the following lemma 
  about the applicative instance of lists:

\includeagda{5}{apcomplete}

  Again, the preconditions of this lemma can be obtained by recursing on the left and 
  right subcodes of the product. 

\subsection{Completeness for Constant Types}

  Since our completeness proof relies on completeness of the generators for constant 
  types, we need the programmer to supply a proof that the supplied generators are 
  indeed complete. To this end, we add a metadata parameter to the type of \ensuremath{\HSVar{deriveGen}}
  -\ensuremath{\HSVar{complete}}, with the following type: 

\includeagda{5}{proofinfotype}

  In order to be able to use the completeness proof from the metadata structure in the 
  \ensuremath{\HSCon{K}} branch of \ensuremath{\HSVar{deriveGen}}-\ensuremath{\HSCon{Complete}}, we need to be able to express the relationship 
  between the metadata structure used in the proof, and the metadata structure used by 
  \ensuremath{\HSVar{deriveGen}}. To do this, we need a way to transform the type of information that is 
  carried by a value of type \ensuremath{\HSCon{KInfo}}: 

\includeagda{5}{kinfomap}

  Given the definition of \ensuremath{\HSCon{KInfo}}-\ensuremath{\HSVar{map}}, we can take the first projection of the 
  metadata input to \ensuremath{\HSVar{deriveGen}}-\ensuremath{\HSCon{Complete}}, and use the resulting structure as input 
  to \ensuremath{\HSVar{deriveGen}}: 

\includeagda{5}{proofinfotype}

  This amounts to the following final type for \ensuremath{\HSVar{deriveGen}}-\ensuremath{\HSCon{Complete}}, where \ensuremath{\HSSym{◂}\HSVar{m}} = \ensuremath{\HSCon{KInfo}}-\ensuremath{\HSVar{map}\;\HSVar{proj₁}\;\HSVar{m}}:  

\includeagda{5}{derivegenwithmd}

  Now, with this explicit relation between the completeness proofs and the generators 
  given to \ensuremath{\HSVar{deriveGen}}, we can simply retrun the proof contained in the metadata of 
  the \ensuremath{\HSCon{K}} branch. 
  
\subsection{Generator Monotonicity}

  The lemma \ensuremath{\HSSym{×}}-\ensuremath{\HSVar{complete}} is not enough to prove completeness in the case of 
  products. We make two recursive calls, that both return a dependent pair with a 
  depth value, and a proof that a value occurs in the enumeration at that depth. 
  However, we need to return just such a dependent pair stating that a pair of both 
  values does occur in the enumeration at a certain depth. The question is what depth 
  to use. The logical choice would be to take the maximum of both dephts. This comes 
  with the problem that we can only combine completeness proofs when they have the 
  same depth value. 

  For this reason, we need a way to transform a proof that some value \ensuremath{\HSVar{x}} occurs in 
  the enumeration at depth \ensuremath{\HSVar{n}} into a proof that \ensuremath{\HSVar{x}} occurs in the enumeration at 
  depth \ensuremath{\HSVar{m}}, given that $n \leq m$. In other words, the set of values that occurs in 
  an enumeration monotoneously increases with the enumeration depth. To finish our 
  completeness proof, this means that we require a proof of the following lemma: 

\includeagda{5}{derivegenmonotone}

  We can complete a proof of this lemma by using the same approach as for the 
  completeness proof. 

\subsection{Final Proof Sketch}

  By bringing all these elements together, we can prove that \ensuremath{\HSVar{deriveGen}} is complete 
  for any code \ensuremath{\HSVar{c}}, given that the programmer is able to provide a suitable 
  metadatastructure. We can transform this proof into a proof that \ensuremath{\HSVar{isoGen}} returns a 
  complete generator by observing that any isomorphism \ensuremath{\HSCon{A}\HSSym{≃}\HSCon{B}} establishes a bijection 
  between the types \ensuremath{\HSCon{A}} and \ensuremath{\HSCon{B}}. Hence, if we apply such an isomorphism to the 
  elements produced by a generator, completeness is preserved. 

  We have the required isomorphism readily at our disposal in \ensuremath{\HSVar{isoGen}}, since it is 
  contained in the instance argument \ensuremath{\HSCon{Regular}\;\HSVar{a}}. This allows us to have \ensuremath{\HSVar{isoGen}} 
  return a completeness proof for the generator it derives: 

\includeagda{5}{isogenproven}

  With which we have shown that if a type is regular, we can derive a complete 
  generator producing elements of that type. 

\chapter{Deriving Generators for Indexed Containers}
  This chapter discusses the universe of \emph{indexed containers} \cite{altenkirch2015indexed}, which provide a generic framework to describe those datatypes that can be defined by induction on their index type. Examples of datatypes we can describe using this universe include finite types \ref{}, vectors \ref{} and well-scoped lambda terms. In this chapter, we give the definition for this universe together with a few examples, and show how a generic generator may be derived for indexed containers. 

\section{Universe Description}

  We choose to follow the representation used by Dagand in \emph{The Essence Of Ornaments} \cite{dagand2017essence}, which provides an excellent introduction to indexed containers, alongside numerous examples. Just as in the previous chapter, we follow the pattern of first defining a datatype describing codes before giving the semantics and fixpoint operation. 

\subsection{Definition}

  Recall our definition of \emph{W-types} in \cref{sec:wtypes}. We purposefully split the canonical definition into three separate definitions for codes, semantics and fixpoint operation. If we consider the datatype describing codes in the universe of indexed descriptions (listing \ref{lst:signatures}), their similarities become clear. Signatures consist of a triple of \emph{operations}, \emph{arities} and \emph{typing discipline} .

\includeagdalisting{6}{signature}{Signatures}{lst:signatures}

  The operations of a signature correspond to a W-type's \emph{shape}, describing the set of available operations. The major difference is that the operations in a signature are parameterized over the index type. Similarly, arity corresponds to position in a W-type, describing the set of recursive subtrees for a given operation. Again, a signature's arity is parameterized over the index type. The typing discipline maps arities to the indices of the corresponding subtrees. 

  The semantics of a signature is, just as for a W-type, a dependent pair, with the first element being a choice of operation, and the second element a function mapping arities to an appropriate recursive type. Contrary to the semantics of a W-type, which maps a code to a value in \ensuremath{\HSCon{Set}\HSSym{\to} \HSCon{Set}}, the semantics of a signature are parameterized over the index type, meaning they map a signature to a value in \ensuremath{\HSSpecial{(}\HSCon{I}\HSSym{\to} \HSCon{Set}\HSSpecial{)}\HSSym{\to} \HSSpecial{(}\HSCon{I}\HSSym{\to} \HSCon{Set}\HSSpecial{)}}. The semantics are shown in listing \ref{lst:sigmtheory}. 

\includeagdalisting{6}{sigmtheory}{The semantics of a signature}{lst:sigmtheory}

  Consequently, the fixpoint operation needs to be lifted from \ensuremath{\HSCon{Set}} to \ensuremath{\HSCon{I}\HSSym{\to} \HSCon{Set}} as well. The required adaptation follows naturally from the definition of the semantics: 

\includeagda{6}{sigfix}

  It is worth noting that, since \ensuremath{\HSCon{Set}\HSSym{≅}\HSSym{⊤}\HSSym{\to} \HSCon{Set}}, we can describe non-indexed datatypes as an indexed container by choosing \ensuremath{\HSSym{⊤}} as the index type. More precisely, there exists a bijection between W-types and signatures indexed with the unit type, such that for every W-type, its interpretation is isomorphic to the interpretation of the corresponding signature, and vice versa. 
  
\subsection{Example Signatures}

  Let us now consider a few examples of datatypes represented as a signature. 

  \begin{example}

    We start by defining a suitable set of operations. The \ensuremath{\HSCon{ℕ}} datatype has two constructor, so we return a type with two inhabitants. We use \ensuremath{\HSSym{⊤}} as the index of the signature, since \ensuremath{\HSCon{ℕ}} is a non-indexed datatype.

\includeagdanv{6}{natop}

    Next, we map each of those operations to the right arity. The \ensuremath{\HSVar{zero}} constructor has no recursive branches, so its arity is the empty type (\ensuremath{\HSSym{⊥}}), while the \ensuremath{\HSVar{suc}} constructor has a single recursive argument, so its arity is the unit type (\ensuremath{\HSSym{⊤}}). 

\includeagdanv{6}{natar}

    Since the index type has only one inhabitant, the associated typing discipline just returns \ensuremath{\HSVar{tt}} in all cases. We bring all these elements together into a single signature, for which we can show that its fixpoint is isomorphic to \ensuremath{\HSCon{ℕ}}.

\includeagdanv{6}{natsig}

  \end{example}

  The signature for natural numbers is quite similar to how we would represent them as a W-type. This example, however, does not tell us much about how signatures enable us to represent indexed datatypes, so let us look at another example. 

  \begin{example}

    We consider the type of finite sets (listing \ref{lst:deffin}). Contrary to natural numbers, the set of available operations varies with different indices. That is, \ensuremath{\HSCon{Fin}\;\HSNumeral{0}} is uninhabited, so the set of operations associated with index \ensuremath{\HSNumeral{0}} is empty. A value of type \ensuremath{\HSCon{Fin}\;\HSSpecial{(}\HSVar{suc}\;\HSVar{n}\HSSpecial{)}} can be constructed using both \ensuremath{\HSVar{suc}} and \ensuremath{\HSVar{zero}}, hence the set of associated operations has two elements: 

\includeagdanv{6}{finop}

    The arity of the \ensuremath{\HSCon{Fin}} type is exactly the same as the arity of \ensuremath{\HSCon{ℕ}}, with the exception of an absurd pattern in the case of index \ensuremath{\HSVar{zero}}. 

\includeagdanv{6}{finar}

    Recall the type of the \ensuremath{\HSVar{suc}} constructor: \ensuremath{\HSCon{Fin}\;\HSVar{n}\HSSym{\to} \HSCon{Fin}\;\HSSpecial{(}\HSVar{suc}\;\HSVar{n}\HSSpecial{)}}. The index of the recursive argument is one less than the index of the constructed value. The typing discipline describes this relation between index of the constructed value, and indices of recursive arguments. In the case of \ensuremath{\HSCon{Fin}}, this means that we map \ensuremath{\HSVar{suc}\;\HSVar{n}} to \ensuremath{\HSVar{n}}, if the index is greater than \ensuremath{\HSNumeral{0}}, and the operation corresponding to the \ensuremath{\HSVar{suc}} constructor is selected. 

\includeagdanv{6}{finty}

    Again, we combine operations, arity and typing into a signature: 

\includeagdanv{6}{finsig}

  \end{example}

  One thing to keep in mind while defining signatures for types is that part of their semantics is a dependent function type. This means that proving an isomorphism between a signature and the type it represents requires some extra work. More specifically, we need to postulate a variation of \emph{extensional equality} for function types: 

\includeagda{6}{funext} 

  One aspect we have not yet addressed is how to represent parameterized types, such as \ensuremath{\HSCon{Vec}\;\HSVar{a}} (listing \ref{lst:defvec}). Indexed containers do not have an explicit way to refer to other types, such as is the case with regular types, but rather include this kind of information as part of a type's operations. 

  \begin{example}

    We consider the \ensuremath{\HSCon{Vec}} type as an example, defining the following operations: 

\includeagdanv{6}{vecop}

    Notice that we map \ensuremath{\HSVar{suc}\;\HSVar{n}} to \ensuremath{\HSCon{A}}, indicating that the \ensuremath{\HSSym{∷}} constructor requires an argument of type \ensuremath{\HSCon{A}}. The remainder of the signature is then quite straightforward: 

\includeagdanv{6}{vecsig}

  \end{example}

\section{Generic Generators for Indexed Containers}

  In order to be able to derive generators from signatures, there are two additional steps we need to take: restricting the set of possible operations and arities, and defining \emph{co-generators} for regular types. 

\subsection{Restricting Operations and Arities}
  
  The set of operations of a signature, \ensuremath{\HSCon{Op}}, is a value in \ensuremath{\HSCon{Set}}. This implies that we have no way to generate values of type \ensuremath{\HSCon{Op}\;\HSVar{i}} without any further input of the programmer. The same problem occurs with arities. We solve this problem by restricting operations and arities to regular types. By doing this, we can reuse the generators we defined for regular types to generate operations and arities. This leads to the slightly altered variation on indexed containers shown in listing (\ref{lst:sigreg}), where \ensuremath{\HSCon{FixR}} and \ensuremath{\HSCon{InR}} denote the fixpoint operation for regular types. The fixpoint operation for signatures remains the same. 

\includeagdalisting{6}{sigreg}{Indexed containers with restricted operations and arities}{lst:sigreg}

  This implies that the definition of signatures changes slightly as well. 

  \begin{example}

    We use the following operation, arity and typing to describe the \ensuremath{\HSCon{Fin}} type as a restricted signature:

\includeagdanv{6}{sigfinreg}

    This definition does not differ too much from the previous one, except that we now pattern match on the fixpoint of some code in \ensuremath{\HSCon{Reg}} instead of directly on the operation or arity. 

  \end{example}

\subsection{Generating Function Types}

  To derive a generator from a signature, we need, in addition to generic generators for regular types, a way to generate function types whose input argument is a regular type. That is, we need to define the following function: 

\includeagda{6}{cogenerate}

  We draw inspiration from SmallCheck's \cite{runciman2008smallcheck} \ensuremath{\HSCon{CoSeries}} typeclass, for which instances can be automatically derived. Co-generators for constant types are to be supplied by a programmer using a metadata structure; we choose to not make this explicit in the type signature. An example definition of \ensuremath{\HSVar{cogenerate}} is included in listing \ref{lst:cogen}.

\includeagdalisting{6}{cogen}{Definition of \ensuremath{\HSVar{cogenerate}}}{lst:cogen}

  Since part of the semantics of an indexed container is a \emph{dependent} function type,  we need to extend \ensuremath{\HSVar{cogenerate}} to work for dependent function types as well. 

\includeagda{6}{picogen}

  The type signature of $\Pi$-\ensuremath{\HSVar{cogenerate}} may look a bit daunting, but it essentially follows the exact same structure as \ensuremath{\HSVar{cogenerate}}. The only real difference is that the the result type of the generated functions may depend on the code we are inducting over, and that we do not take a generator as input, but rather a function from index to generator. The definitions of $\Pi$-\ensuremath{\HSVar{cogenerate}} and \ensuremath{\HSVar{cogenerate}} are virtually the same, but we need to make the dependency between argument and result type explicit in the type in order for Agda to be able to solve all metavariables. 

\subsection{Constructing the Generator}

  We are now ready to construct a the generic generator for indexed descriptions. Recall that \ensuremath{\HSVar{deriveGen}\;\HSVar{r}\;\HSVar{r}} returns a generator for the regular type represented by r. 

\includeagda{6}{gensignature}

  The final generator is quite simple, really. We use the existing functionality for regular types to generate operations and arities, and return them as a dependent pair, wrapping and unwrapping fixpoint operations as we go along. The dependency between the first and second element of said pair is captured using by using the monadic structure of the generator type.

  Unfortunately, we have not been able to assemble a completeness proof for the enumeration derived using $\Sigma$-\ensuremath{\HSVar{generate}}. As was the case with the completeness proof for regular types, we need to explicitly pattern match on the value for which we are proving that it occurs in the enumeration in order for the termination checker to recognize that the proof can be constructed in finite time. However, since part of the semantics of a signature is a function type, we would require induction over function types in order to complete the proof. 


\chapter{Deriving Generators for Indexed Descriptions}

We use the generic description for indexed datatypes proposed by Dagand \cite{dagand2013cosmology} in his PhD thesis. First, we give the definition and semantics of this universe, before showing how a generator can be derived from codes in this universe. Finally, we prove that the enumerations resulting from these generators are complete. 

\section{Universe Description}

  

\subsection{Definition}\label{sec:idescdef}

  Indexed descriptions are not much unlike the codes used to describe regular types (that is, the \ensuremath{\HSCon{Reg}} datatype), with the differences being: 

\begin{enumerate}
  \item 
  A type parameter \ensuremath{\HSCon{I}\HSCon{\mathbin{:}}\HSCon{Set}}, describing the type of indices.

  \item 
  A generalized coproduct, \ensuremath{\HSSpecial{`}}$\sigma$, that denotes choice between $n$ constructors, in favor of the \ensuremath{\HSSym{⊕}} combinator. 

  \item 
  A combinator denoting dependent pairs. 

  \item 
  Recursive positions storing the index of recursive values. 
\end{enumerate}

  This amounts to the Agda datatype describing indexed descriptions shown in listing \ref{lst:idesc}. 

\includeagdalisting{7}{idesc}{The Universe of indexed descriptions}{lst:idesc}

  Notice how we retain the regular product of codes as a first order construct in our universe. The \ensuremath{\HSCon{Sl}} datatype is used to select the right branch from the generic coproduct, and is isomorphic to the \ensuremath{\HSCon{Fin}} datatype. 

\includeagda{7}{sl}

  The semantics associated with the \ensuremath{\HSCon{IDesc}} universe is largely the same as the semantics of the universe of regular types. The key difference is that we do not map codes to a functor \ensuremath{\HSCon{Set}\HSSym{→}\HSCon{Set}}, but rather to \ensuremath{\HSCon{IDesc}\;\HSCon{I}\HSSym{→}\HSSpecial{(}\HSCon{I}\HSSym{→}\HSCon{Set}\HSSpecial{)}\HSSym{→}\HSCon{Set}}. The semantics is shown in listing \ref{lst:idescsem}.

\includeagdalisting{7}{idescsem}{Semantics of the IDesc universe}{lst:idescsem}

  We do not require a separate constructor representing the empty type, as we can encode it as a coproduct over zero constructors: \ensuremath{\HSSpecial{`}}$\sigma$ \ensuremath{\HSNumeral{0}} $\lambda$ \ensuremath{\HSSpecial{(}\HSSpecial{)}}. 

  We calculate the fixpoint of interpreted codes using the following fixpoint combinator: 

\includeagda{7}{idescfix}

  \begin{example}
    We can describe the \ensuremath{\HSCon{Fin}} datatype, listing \ref{lst:deffin}, as follows using a code in the universe of indexed descriptions: 

\includeagdanv{7}{idescfin}

    If the index is \ensuremath{\HSVar{zero}}, there are no inhabitants, so we return a coproduct of zero choices. Otherwise, we may choose between two constructors: \ensuremath{\HSVar{zero}} or \ensuremath{\HSVar{suc}}. Notice that we describe the datatype by induction on the index type. This is not required, althoug convenient in this case. A different, but equally valid description exists, in which we use the \ensuremath{\HSSpecial{`}}$\Sigma$ constructor to explicitly enforce the constraint that the index \ensuremath{\HSVar{n}} is the successor of some \ensuremath{\HSVar{n'}}. 
    
\includeagdanv{7}{idescfin2}
    
    Listing \ref{lst:finiso} establishes that the fixpoint of \ensuremath{\HSVar{finD}} is indeed isomorphic to \ensuremath{\HSCon{Fin}}. 

  \end{example}

\includeagdalisting{7}{idescfiniso}{Isomorphism between \ensuremath{\HSCon{Fix}\;\HSVar{finD}} and \ensuremath{\HSCon{Fin}}}{lst:finiso}

\subsection{Exmample: describing well typed lambda terms}

  To demonstrate the expressiveness of the \ensuremath{\HSCon{IDesc}} universe, and to show how one might model a more complex datatype, we consider simply typed lambda terms as an example. We assume raw terms as described in listing \ref{lst:defrawterm}. We type terms using the universe described in listing \ref{lst:defstype}. 

\subsubsection{Modelling SLC in Agda}

  We write $\Gamma \ni \alpha : \tau$ to signify that $\alpha$ has type $\tau$ in $\Gamma$. Context membership is described by the following inference rules: 

\begin{equation*}
\texttt{[Top]}
  \frac{}{\Gamma , \alpha : \tau \ni \alpha : \tau} \quad 
\texttt{[Pop]}
  \frac{\Gamma \ni \alpha : \tau}{\Gamma , \beta : \sigma \ni \alpha : \tau}
\end{equation*}

  We describe these inference rules in Agda using an inductive datatype, shown in listing \ref{lst:ctxmem}, indexed with a type and a context, whose inhabitants correspond to all proofs that a context $\Gamma$ contains a variable of type $\tau$. 

\includeagdalisting{7}{ctxmembership}{Context membership in Agda}{lst:ctxmem}

  We write $\Gamma \vdash t : \tau$ to express a typing judgement stating that term $t$ has type $\tau$ when evaluated under context $\Gamma$. The following inference rules determine when a term is type correct: 

\begin{equation*}
\texttt{[Var]}
  \frac{\Gamma \ni \alpha : \tau}{\Gamma \vdash \alpha : \tau} \quad 
\texttt{[Abs]}
  \frac{\Gamma , \alpha : \sigma \vdash t : \tau}{\Gamma \vdash \lambda\ \alpha\ .\ t : \sigma \rightarrow \tau} \quad
\texttt{[App]}
  \frac{\Gamma \vdash t1 : \sigma \rightarrow \tau \quad \Gamma \vdash t2 : \sigma}{\Gamma \vdash t_1\ t_2 : \tau}
\end{equation*} 

  We model these inference rules in Agda using a two way relation between contexts and types whose inhabitants correspond to all terms that have a given type under a given context (listing \ref{lst:wflambda})

\includeagdalisting{7}{typejudgement}{Well-typed lambda terms as a two way relation}{lst:wflambda}

  Given an inhabitant $\Gamma$ \ensuremath{\HSSym{⊢}} $\tau$ of this relationship, we can write a function \ensuremath{\HSVar{toTerm}} that transforms the typing judgement to its corresponding untyped term. 

\includeagda{7}{toterm}

  The term returned by \ensuremath{\HSVar{toTerm}} will has type $\tau$ under context $\Gamma$. 

\subsubsection{Describing well typed terms}

  In \cref{sec:idescdef}, we saw that we can describe the \ensuremath{\HSCon{Fin}} both by induction on the index, as well as by adding explicit constraints. Similarly, we can choose to define a description in two ways: either by induction on the type of the terms we are describing, or by including an explicit constraint that the index type is a function type for the description of the abstraction rule. If we consider a description for lambda terms using induction on the index (listing \ref{slcdescinductive}), we see that it has a downside. The same constructor may yield a value with different index patterns. 

\includeagdalisting{7}{slcdescinductive}{A description for well typed terms using induction on the index type}{lst:slcdescinductive}

  For example, the application rule may yield both a function type as well as a ground type, we need to include a description for this constructor in both branches when pattern matching on the input type. If we compare the inductive description to a version that explicitly includes a constraint that the input type is a function type in case of the description for the abstraction rule, we end up with a much more succinct description. 

  However, using such a description comes at a price. The descriptions used will become more complex, hence their interpretation will too. Additionally, we delay the point at which it becomes apparent that a constructor could not have been used to create a value with the input index. This makes the generators for indexed descriptions, which we will derive in the next section, potentially more computationally intensive to run when derived from a description that uses explicit constraints, compared to an equivalent description that is defined by induction on the index. 

\includeagdalisting{7}{slcdescconstrained}{A description for well typed terms using explicit constraints}{lst:slcdescconstrained} 

  To convince ourselves that these descriptions do indeed describe the same type, we can show that their fixpoints are isomorphic: 

\includeagda{7}{desciso}

  Given an isomorphism between the fixpoints of two descriptions, we can prove that they are both isomorphic to the target type by establishing an isomorphism between the fixpoint of one of them and the type we are describing. For example, we might prove the following isomorphism: 

\includeagda{7}{constrainediso}

  Using the transitivity of \ensuremath{\HSSym{\anonymous} \HSSym{≃\char95 }}, we can show that the inductive description also describes well typed terms. 

\section{Generic Generators for Indexed Descriptions}



\section{Completeness Proof for Generators Derived From Indexed Descriptions}

\chapter{Program Term Generation}

\chapter{Implementation in Haskell}

\chapter{Conclusion \& Further Work}

\appendix
\chapter{Datatype Definitions}

\section{Natural numbers}

\begin{listing}{Definition of natural numbers in Haskell and Agda}{lst:defnat}

  \begin{hscode}\SaveRestoreHook
\column{B}{@{}>{\hspre}l<{\hspost}@{}}%
\column{5}{@{}>{\hspre}l<{\hspost}@{}}%
\column{15}{@{}>{\hspre}c<{\hspost}@{}}%
\column{15E}{@{}l@{}}%
\column{18}{@{}>{\hspre}l<{\hspost}@{}}%
\column{E}{@{}>{\hspre}l<{\hspost}@{}}%
\>[5]{}\HSKeyword{data}\;\HSCon{Nat}{}\<[15]%
\>[15]{}\HSSym{\mathrel{=}}{}\<[15E]%
\>[18]{}\HSCon{Zero}{}\<[E]%
\\
\>[15]{}\HSSym{\mid} {}\<[15E]%
\>[18]{}\HSCon{Suc}\;\HSCon{N}{}\<[E]%
\ColumnHook
\end{hscode}\resethooks

  \dotfill

  \appincludeagda{A}{nat}

\end{listing}

\section{Finite Sets}

\begin{listing}{Definition of finite sets in Agda}{lst:deffin}

  \appincludeagda{A}{fin}

\end{listing}
\newpage
\section{Vectors}

\begin{listing}{Definition of vectors (size-indexed listst) in Agda}{lst:defvec}

  \appincludeagda{A}{vec}

\end{listing}

\section{Simple Types}

\begin{listing}{Definition of simple types in Haskell and Agda}{lst:defstype}

  \begin{hscode}\SaveRestoreHook
\column{B}{@{}>{\hspre}l<{\hspost}@{}}%
\column{3}{@{}>{\hspre}l<{\hspost}@{}}%
\column{14}{@{}>{\hspre}c<{\hspost}@{}}%
\column{14E}{@{}l@{}}%
\column{17}{@{}>{\hspre}l<{\hspost}@{}}%
\column{E}{@{}>{\hspre}l<{\hspost}@{}}%
\>[3]{}\HSKeyword{data}\;\HSCon{Type}{}\<[14]%
\>[14]{}\HSSym{\mathrel{=}}{}\<[14E]%
\>[17]{}\HSCon{T}{}\<[E]%
\\
\>[14]{}\HSSym{\mid} {}\<[14E]%
\>[17]{}\HSCon{Type}\HSSym{:->:}\HSCon{Type}{}\<[E]%
\ColumnHook
\end{hscode}\resethooks

  \dotfill

  \appincludeagda{A}{simpletypes}

\end{listing}

\section{Contexts}

\begin{listing}{Definition of contexts in Haskell and Agda}{lst:defcontext}

  \begin{hscode}\SaveRestoreHook
\column{B}{@{}>{\hspre}l<{\hspost}@{}}%
\column{3}{@{}>{\hspre}l<{\hspost}@{}}%
\column{13}{@{}>{\hspre}c<{\hspost}@{}}%
\column{13E}{@{}l@{}}%
\column{16}{@{}>{\hspre}l<{\hspost}@{}}%
\column{E}{@{}>{\hspre}l<{\hspost}@{}}%
\>[3]{}\HSKeyword{data}\;\HSCon{Ctx}{}\<[13]%
\>[13]{}\HSSym{\mathrel{=}}{}\<[13E]%
\>[16]{}\HSCon{Empty}{}\<[E]%
\\
\>[13]{}\HSSym{\mid} {}\<[13E]%
\>[16]{}\HSCon{Cons}\;\HSCon{Id}\;\HSCon{Type}\;\HSCon{Ctx}{}\<[E]%
\ColumnHook
\end{hscode}\resethooks

  \dotfill

  \appincludeagda{A}{context}

\end{listing}
\newpage
\section{Raw Lambda Terms}

\begin{listing}{Definition of raw lambda terms in Haskell and Agda}{lst:defrawterm}

  \begin{hscode}\SaveRestoreHook
\column{B}{@{}>{\hspre}l<{\hspost}@{}}%
\column{3}{@{}>{\hspre}l<{\hspost}@{}}%
\column{12}{@{}>{\hspre}c<{\hspost}@{}}%
\column{12E}{@{}l@{}}%
\column{15}{@{}>{\hspre}l<{\hspost}@{}}%
\column{E}{@{}>{\hspre}l<{\hspost}@{}}%
\>[3]{}\HSKeyword{data}\;\HSCon{RT}{}\<[12]%
\>[12]{}\HSSym{\mathrel{=}}{}\<[12E]%
\>[15]{}\HSCon{Var}\;\HSCon{Id}{}\<[E]%
\\
\>[12]{}\HSSym{\mid} {}\<[12E]%
\>[15]{}\HSCon{Abs}\;\HSCon{Id}\;\HSCon{RT}{}\<[E]%
\\
\>[12]{}\HSSym{\mid} {}\<[12E]%
\>[15]{}\HSCon{App}\;\HSCon{RT}\;\HSCon{RT}{}\<[E]%
\ColumnHook
\end{hscode}\resethooks

  \dotfill

  \appincludeagda{A}{rawterm}

\end{listing}

\section{Lists}

\begin{listing}{Definition lists and Agda}{lst:deflist}

  \appincludeagda{A}{list}

\end{listing}

\section{Well-scoped Lambda Terms}

\begin{listing}{Definition well-scoped lambda terms in Agda}{lst:defwellscoped}

  \appincludeagda{A}{wellscoped}

\end{listing}

\backmatter
\listoffigures
\listoftables

\bibliographystyle{acm}
\bibliography{references}

\end{document}


